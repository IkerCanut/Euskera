\documentclass[12pt, a4paper, landscape]{article}
\usepackage[left=2.5cm, right=2.5cm, top=3.5cm, bottom=2cm]{geometry}
\usepackage{multirow}
\usepackage{multicol}
\begin{document}
\author{Iker M. Canut}
\title{A2\\
\large{Resumen de Euskera\\}
}
\maketitle
\newpage

\tableofcontents
\newpage

\section{Aditzak}
\begin{center}
\begin{table}[h]
\begin{tabular}{c | c | c}
EGIN & EGITEN & (hacer) \\
IRTEN & IRTETEN & (salir) \\
SARTU & SARTZEN & (entrar) \\
EDAN & EDATEN & (beber) \\
JAN & JATEN & (comer) \\
IKUSI & IKUSTEN & (ver) \\
PRESTATU & PRESTATZEN & (preparar) \\
MOZTU & MOZTEN & (cortar) \\
IPINI & IPINTZEN & (poner) \\
ZURITU & ZURITZEN & (pelar) \\
GARBITU & GARBITZEN & (limpiar) \\
DEITU & DEITZEN & (llamar) \\
IDATZI & IDATZEN & (escribir) \\
BAZKALDU & BAZKALTZEN & (comer, almorzar) \\
AFALDU & AFALTZEN & (cenar) \\
\end{tabular}
\end{table}
\end{center}

\newpage

\section{Eguraldia}
% \begin{multicols}{2}
\begin{itemize}
\item PROZEZUAK
\begin{itemize}
\item Haizea egin
\item Zirimiria egin
\item Euria egin
\item Kazkabarra egin
\item Elurra egin
\end{itemize}
\item EGOERAK
\begin{itemize}
\item Eguraldi ona egin
\item Eguraldi txarra egin
\item Hotz egin
\item Bero egin
\item Fresko egin
\item Sargori egin
\end{itemize}
\end{itemize}
% \end{multicols}

\begin{center}
\begin{table}[h]
\begin{tabular}{|c|c|c|}
\hline
& PROZEZUAK & EGOERAK \\
\hline
ORAIN & Ari du & Egiten du \\
GAUR & Egin du & Egin du \\
BIHAR & Egingo du & Egingo du \\
\hline
\end{tabular}
\end{table}
\end{center}

\end{document}