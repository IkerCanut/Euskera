\documentclass[11pt, a4paper]{article}
\usepackage[left=2cm, right=1.5cm, top=2cm, bottom=-8cm]{geometry}
\usepackage{multirow}
\begin{document}
% \author{Iker M. Canut}
\date{}
\title{Verbos Sintéticos\\}
\maketitle

Los verbos sintéticos son aquellos que se pueden inflexionar sin la ayuda de un verbo auxiliar. Reúnen toda la información del verbo en una palabra: persona, significado de la acción, modo, tiempo y aspecto. En euskera hay hasta 28 verbos sintéticos, de los cuales son 9 los que se utilizan mas comúnmente. En castellano podríamos decir que todos los verbos son sintéticos, es decir \textit{voy} implica que soy \textit{yo} y que es \textit{ahora} cuando lo hago, de la misma manera \textit{tuvieron} implica que fueron \textit{ellos} y que es \textit{pasado}.\\

Hay que tener en cuenta que cuando hablamos de la forma genérica de NOR-NORI lo estamos haciendo de la forma NOR-NORI del verbo sintético IZAN, cuando hablamos de NOR-NORK y ZER-NORI-NORK lo hacemos de la forma sintética del verbo UKAN, es decir, usamos como auxiliares las formas sintéticas de esos verbos.\\

Un ejemplo, para decir \textit{Se lo llevo} (Yo a él) en Euskera se podría usar \textit{Eramaten diot}, que es lo mas frecuente (Verbo+Auxiliar). Aunque también se podría usar su forma ZER-NORI-NORK sintética: \textit{Daramakiot}.\\

Para recordar:
\begin{itemize}
	\item Casi todo en euskera en pasado acaba en N. Lo mismo aplica a las formas sintéticas.
	\item Existe un prefijo para cada persona: \textit{N} para \textit{Yo}, \textit{H} para \textit{tu}, \textit{Z} para \textit{usted} y \textit{vosotros}, \textit{D} para \textit{él, ella y ellos} y \textit{G} para \textit{Nosotros}.\\
\end{itemize}

En las tablas que están a continuación, se usan las siguientes convenciones:
\begin{itemize}
	\item La marca \textit{[k/n]} hace referencia al género de la persona en cuestión.
	\item Los verbos que tienen una marca entre paréntesis hacen referencia al plural. Es decir, es lo que se añade cuando la acción recae sobre varias cosas.
	\item Las primeras 7 filas corresponden al tiempo presente y el resto al pasado.
\end{itemize}

Únicamente se nombran los mas conocidos y algunos ejemplos.
\newpage

\begin{table}[h]
	\centering
	\begin{tabular}{|c|c|c|c|c|}
		\hline
		\textbf{ IZAN }  & \textbf{ EGON }   & \textbf{ETORRI}   & \textbf{JOAN}  & \textbf{IBILI}    \\
		\textbf{ (ser) } & \textbf{(estar) } & \textbf{(venir) } & \textbf{(ir) } & \textbf{(andar) } \\ \hline\hline
		Naiz             & Nago              & Nator             & Noa            & Nabil             \\ \hline
		Haiz             & Hago              & Hator             & Hoa            & Habil             \\ \hline
		Da               & Dago              & Datoz             & Doa            & Dabil             \\ \hline
		Gara             & Gaude             & Gatoz             & Goaz           & Gabiltza          \\ \hline
		Zara             & Zaude             & Zatoz             & Zoaz           & Zabiltza          \\ \hline
		Zarete           & Zaudete           & Zatozte           & Zoazte         & Zabiltzate        \\ \hline
		Dira             & Daude             & Datozte           & Doazte         & Dabiltza          \\ \hline\hline
		Nintzen          & Nengoen           & Nentorren         & Nihoan         & Nenbilen          \\ \hline
		Hintzen          & Hengoen           & Hentorren         & Hindoan        & Henbilen          \\ \hline
		Zen              & Zegoen            & Zetorren          & Zihoan         & Zebilen           \\ \hline
		Ginen            & Geunden           & Gentozen          & Gindoazen      & Genbiltzan        \\ \hline
		Zinen            & Zeunden           & Zentozen          & Zindoazen      & Zenbiltzan        \\ \hline
		Zineten          & Zeundeten         & Zentozten         & Zindoazten     & Zenbiltzaten      \\ \hline
		Ziren            & Zeuden            & Zetozen           & Zihoazen       & Zebiltzan         \\ \hline
	\end{tabular}
\end{table}

\begin{table}[h]
	\centering
	\begin{tabular}{|c|c|c|c|}
		\hline
		\textbf{EDUKI}   & \textbf{JAKIN}   & \textbf{ERAMAN}   & \textbf{EKARRI}            \\
		\textbf{(tener)} & \textbf{(saber)} & \textbf{(llevar)} & \textbf{(traer)}           \\ \hline\hline
		Dau(z)kat        & Daki(zki)t       & Darama(tza)t      & Dakar(tza)t                \\ \hline
		Dau(z)ka[k/n]    & Daki(zki)[k/n]   & Darama(tza)[k/n]  & Dakar(tza)[k/n]            \\ \hline
		Dau(z)ka         & Daki(zki)        & Darama(tza)       & Dakar(tza)                 \\ \hline
		Dau(z)kagu       & Daki(zki)gu      & Darama(tza)gu     & Dakar(tza)gu               \\ \hline
		Dau(z)kazu       & Daki(zki)zu      & Darama(tza)zu     & Dakar(tza)zu               \\ \hline
		Dau(z)kazue      & Daki(zki)zue     & Darama(tza)zue    & Dakar(tza)zue              \\ \hline
		Dau(z)kate       & Daki(zki)te      & Darama(tza)te     & Dakar(tza)te               \\ \hline\hline
		Neu(z)kan        & Neki(zki)en      & Nerama(tza)n      & Nekarren/(Nekartzan)       \\ \hline
		Heu(z)kan        & Heki(zki)en      & Herama(tza)n      & Hekarren/(Hekartzan)       \\ \hline
		Zeuz(z)kan       & Zeki(zki)en      & Zerama(tza)n      & Zekarren/(Zekartzan)       \\ \hline
		Geneu(z)kan      & Geneki(zki)en    & Generama(tza)n    & Genekarren/(Genekartzan)   \\ \hline
		Zeneu(z)kan      & Zeneki(zki)en    & Zenerama(tza)n    & Zenekarren/(Zenekartzan)   \\ \hline
		Zeneu(z)katen    & Zeneki(zki)eten  & Zenerama(tza)ten  & Zenekarten/(Zenekartzaten) \\ \hline
		Zeu(z)katen      & Zeki(zki)ten     & Zerama(tza)ten    & Zekarten/(Zekartzaten)     \\ \hline
	\end{tabular}
\end{table}

\begin{table}[h]
	\centering
	\begin{tabular}{|c|c|c|}
		\hline
		\textbf{ESAN}    & \textbf{IRUDITU}   & \textbf{IRITZI}   \\
		\textbf{(decir)} & \textbf{(parecer)} & \textbf{(opinar)} \\ \hline\hline
		Diot             & Dirudit            & Deritzot          \\ \hline
		Dio[k/n]         & Dirudi[k/n]        & Deritzo[k/n]      \\ \hline
		Dio              & Dirudi             & Deritzo           \\ \hline
		Diogu            & Dirudigu           & Deritzogu         \\ \hline
		Diozu            & Dirudizu           & Deritzozu         \\ \hline
		Diozue           & Dirudizue          & Deritzozue        \\ \hline
		Diote            & Dirudite           & Deritzote         \\ \hline\hline
		Nioen            & Nirudien           & Neritzon          \\ \hline
		Hioen            & Hirudien           & Heritzon          \\ \hline
		Zioen            & Zirudien           & Zeritzon          \\ \hline
		Genioen          & Genirudien         & Generitzon        \\ \hline
		Zenioen          & Zenirudien         & Zeneritzon        \\ \hline
		Zenioten         & Zenirudieten       & Zeneritzoten      \\ \hline
		Zioten           & Ziruditen          & Zeritzoten        \\ \hline
	\end{tabular}
\end{table}
\end{document}
