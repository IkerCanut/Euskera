\documentclass[11pt, a4paper]{article}
\usepackage[left=2cm,right=2cm,top=2cm,bottom=2cm]{geometry}
\usepackage{multirow}
\usepackage{multicol}
\begin{document}
\author{Iker M. Canut}
\title{Resumen de Euskera\\}
\maketitle
\newpage

\section{A1 - Ni Amaia naiz}
\subsection{El Alfabeto}
C, Q, V, W  e  Y no son letras del alfabeto del euskera. Son letras que sólo se utilizan con nombres de otras lenguas.

\subsection{La pronunciación de s / ts, z /tz y x / tx: }
El modo de articulación en euskera de los sonidos s, z y x es parecido al de la s castellana. En cambio, el modo de articulación de los sonidos ts, tz y tx es parecido al de la ch castellana.

Por otro lado, para producir los sonidos s y ts hay que colocar la punta de la lengua en el paladar. En cambio, para crear los sonidos z y tz hay que colocar la punta de la lengua en la base de los dientes superiores. Por último, los sonidos x y tx se producen colocando el cuerpo de la lengua en el paladar.


\subsection{La pronunciación de g y j:}
La g en euskera se debe pronunciar de forma suave, igual que la g de la palabra castellana gota. En euskera, al contrario que en castellano, no se debe poner una u delante de las letras i y e.

En cuanto a la letra j, hay dos pronunciaciones posibles en euskera: una parecida a la pronunciación de la letra y de la palabra castellana ayuda; la otra, parecida a la pronunciación de la j de la palabra castellana junta. Las dos pronunciaciones son correctas, pero se tiene por mejor la primera, por ser la original del euskera. 

\subsection{La pronunciación de l y n tras i: }
En algunos dialectos del euskera, la l y n que van tras una i se pronuncian ll y ñ respectivamente.

\subsection{La pronunciación de s y ts tras i:}
Pasa algo parecido con los sonidos s y ts. En algunos dialectos se tiende a la palatalización de estos sonidos tras i, y en consecuencia, se pronuncian x y tx.

\subsection{La pronunciación de la z final en uniones de palabras:}
La letra z sufre numerosas alteraciones fonéticas en el euskera hablado. Estas alteraciones tienen lugar sobre todo con la partícula negativa EZ y con numerosas formas verbales. He aquí estas alteraciones: 
\begin{itemize}
	\item \textbf{Z + N $>$ N}: ez naiz se pronuncia \textbf{enaiz}.
	\item \textbf{Z + L $>$ L}: ez litzateke se pronuncia \textbf{elitzateke}.
	\item \textbf{Z + H $>$ $Ø$}: ez haiz se pronuncia \textbf{eaiz}.
	\item \textbf{Z + Z $>$ TZ}: ez zara se pronuncia \textbf{etzara}.
	\item \textbf{Z + D $>$ ZT}: ez da se pronuncia \textbf{ezta}.
	\item \textbf{Z + G $>$ ZK}: ez gara se pronuncia \textbf{ezkara}.
	\item \textbf{Z + B- $>$ ZP}: ez bada se pronuncia \textbf{ezpada}.
\end{itemize}

\subsection{El acento}
En euskera se tiende a acentuar la segunda sílaba empezando por la izquierda. La última sílaba también se acentúa ligeramente.

\subsection{Deklinabidea}
Igual que existen preposiciones en castellano, en euskera hay posposiciones, sufijos que suelen ir pegados a los nombres.

\subsection{El Verbo}
En euskera las formas verbales pueden ser \textbf{sintéticas} (una palabra) o \textbf{perifrásticas} (dos palabras). En las formas verbales sintéticas, toda la información verbal se concentra en una sola forma. Las formas verbales perifrásticas en cambio se dividen en más de un elemento: \textbf{verbo principal} + \textbf{verbo auxiliar}. El verbo principal puede tomar cuatro formas, por ejemplo:
\begin{itemize}
\item EROS: La raiz del verbo. Sagar bat eros ezazu!
\item EROSI: El participio. Sagar bat erosi dut.
\item EROSTEN: El imperfecto. Egunero sagar bat erosten dut.
\item EROSIKO: El futuro. Bihar sagar bat erosiko dut.
\end{itemize}
En cambio, el verbo auxiliar nos informa acerca de la persona o personas que hacen la acción que indica el verbo [nos da información sobre NOR (¿qué?), NORI (¿a quién?) y NORK (¿quién?)], el tiempo y el modo en el que transcurre dicha acción.

\subsection{Pronombres Personales}
\begin{itemize}
\begin{multicols}{2}
\item \textbf{Ni}:  1ª persona del singular
\item \textbf{Hi}:  2ª persona del singular (informal)
\item \textbf{Zu}:  2ª persona del singular (formal)
\item \textbf{Hura}:  3ª persona del singular
\item \textbf{Gu}:  1ª persona del plural
\item \textbf{Zuek}:  2ª persona del plural
\item \textbf{Haiek}:  3ª persona del plural
\end{multicols}
\end{itemize}

\subsection{Orden habitual en euskera}
\indent \indent \textbf{Sujeto + Complemento + Verbo}\\

De todas maneras, el elemento principal, el elemento inquirido o \textbf{GALDEGAIA}, se coloca delante del verbo. Por lo tanto, en euskera los elementos de la oración se ordenan según la importancia que tengan, colocándose el galdegaia delante del verbo. Los elementos restantes, si son importantes, se colocan al principio de la oración; en cambio, si no lo son, se colocan al final.\\

En las oraciones negativas se ha de intercalar una partícula que no aparece en las afirmativas: EZ (no). En consecuencia, esto afecta al orden de los elementos de la oración.\\
\indent \textbf{Sujeto EZ Verbo (Complemento)}\\

Si el verbo utilizado es perifrástico:\\
\indent \textbf{Sujeto EZ Auxiliar (Complemento) Verbo}

\subsection{IZAN}
\begin{itemize}
\begin{multicols}{2}
\item Ni naiz (yo soy)
\item Hi haiz (tú eres)
\item Hura da (él es)\\
\item Gu gara (nosotros somos)
\item Zu zara (tú eres)
\item Zuek zarete (vosotros sois)
\item Haiek dira (ellos son)
\end{multicols}
\end{itemize}

\section{A1 - Auto hau berria da}
En euskera, el \textbf{artículo determinado} tiene dos formas: \textbf{-A} (singular) y \textbf{-AK} (plural).\\
etxeA (la casa), etxeAK (las casas), etxe handiA (la casa grande), etxe handiAK (las casas grandes).\\
Nota: si el nombre termina en -a, NO se le añade otra -a.\\
Nota: los nombres propios NO llevan artículo.\\

\textbf{BAT} (uno / una) y \textbf{BATZUK} (unos / unas) son \textbf{artículos indeterminados}.\\
etxe BAT (una casa), etxe BATZUK (unas casas),\\ etxe handi BAT (una casa grande), etxe handi BATZUK (unas casas grandes).

\subsection{Pronombres Demostrativos}
Los pronombres demostrativos se colocan detrás del nombre o adjetivo
\begin{itemize}
\item hau (este, esta, esto)
\item hori (ese, esa, eso)
\item hura (aquel, aquella, aquello)
\item hauek (estos, estas)
\item horiek (esos, esas)
\item haiek (aquellos, aquellas)
\end{itemize}
etxe HAU (esta casa), etxe HAUEK (estas casas),\\ etxe handi HAU (esta casa grande), etxe handi HAUEK (estas casas grandes).\\

Si el sujeto es singular, el adjetivo llevará el artículo singular. Si es plural, el plural.\\
EtxeA handiA da (La casa es grande), Liburu HAUEK interesgarriAK dira (Estos libros son interesantes).

\subsection{Uso Correcto de los Articulos}
\textbf{EtxeA} handiA da (La casa es grande).\\
\textbf{Etxe handiA} berriA da (La casa grande es nueva).\\
En el segundo ejemplo, el sujeto está formado por un nombre y un adjetivo: \textit{etxe} y \textit{handi} forman una unidad. Cuando ésto es así, el artículo se coloca al final de la unidad.

\subsection{Algunos Adjetivos}
\begin{itemize}
\begin{multicols}{2}
\item interesante=interesgarria
\item aburrido/a=aspergarria
\item bonito/a=polita
\item feo/a=itsusia
\item rápido/a=azkarra
\item lento/a=mantsoa
\item alto/a=altua
\item bajo/a=baxua
\item grande=handia
\item pequeño/a=txikia
\item nuevo/a=berria
\item viejo/a=zaharra
\item buen/o/a=ona
\item malo/a=gaiztoa
\item largo/a=luzea
\item corto/a=motza
\item obeso/a=gizena
\item delgado/a=argala
\item listo/a=argia
\item tonto/a=tentela
\end{multicols}
\end{itemize}

\section{A1 - Nor zara zu? Ni Xabi naiz, Zarauzkoa}
\textbf{NOR?} (¿Quién? / ¿Quiénes?), \textbf{ZER?} (¿Qué?).\\
\indent Zer da hau? Hau mahaia da (¿Qué es ésto? Esto es la mesa).\\
\indent Nor zarete zuek? Gu Xabi eta Idoia gara (¿Quiénes sois vosotros? Nosotros somos Xabi e Idoia).\\

\noindent Para preguntar sobre profesiones también se utiliza el interrogativo \textbf{ZER}.\\
\indent Zer da Idoia? Idoia enpresaburua da (¿Qué es Idoia? Idoia es empresaria).\\

\noindent Cuando se quiere preguntar sobre la procedencia de alguien o algo se utiliza el interrogativo \textbf{NONGO}. En la respuesta se le añade \textbf{-koa/-koak} al nombre propio de lugar.\\
\indent Nongoa da Amaia? Amaia Oñatikoa da (¿De dónde es Amaia? Amaia es de Oñati).\\
\indent Nongoak dira politikari horiek? Politikari horiek Baionakoak dira (¿De dónde son esos políticos? Esos políticos son de Baiona).

\subsection{Agurrak (saludos)}
\begin{itemize}
\begin{multicols}{2}
\item Kaixo!: ¡Hola!
\item Aspaldiko!: ¡Cuánto tiempo sin verte!
\item Egun on!: ¡Buenos días!
\item Arratsalde on!: ¡Buenas tardes!
\item Gabon!: ¡Buenas noches!
\item Urte askotarako!: ¡Encantado!
\item Agur!: ¡Adiós!
\item Gero arte!: ¡Hasta luego!
\item Bihar arte!: ¡Hasta mañana!
\item Ikusi arte!: ¡Hasta la vista!
\end{multicols}
\item Zer moduz? Ongi, eta zu?: ¿Qué tal? Bien, ¿y tú?
\end{itemize}

\subsection{Profesiones}
\begin{itemize}
\begin{multicols}{2}
\item Arrantzalea: pescador
\item Irakaslea: profesor
\item Futbolaria: futbolista
\item Arotza: carpintero
\item Iturgina: fontanero
\item Etxekoandrea: ama de casa
\item Postaria: cartero
\item Ingeniaria: ingeniero
\item Enpresaburua: empresario
\item Alkatea: alcalde
\item Sendagilea: médico
\end{multicols}
\end{itemize}

\section{A1 - Nolakoa da autoa?}
El interrogativo \textbf{NOLAKO} se utiliza para preguntar cómo es algo.\\
\indent NolakoA da etxeA ? EtxeA handiA da (¿Cómo es la casa? La casa es grande).\\
\indent NolakoAK dira autoAK ? AutoAK azkarrAK dira (¿Cómo son los coches? Los coches son rápidos).\\
En las respuestas, los nombres y adjetivos llevan la misma marca.\\

\noindent La conjunción \textbf{BAINA} (pero) indica oposición:\\
\indent Liburua interesgarria da, baina motza (El libro es interesante pero corto).\\

\noindent Para preguntar sobre el color de algo se utiliza la siguiente pregunta: \textbf{Zein koloretakoa da … ?} (¿De qué color es?).\\
\indent  Zein koloretakoa da zerua? Zerua urdina da. (¿De qué color es el cielo? El cielo es azul.)\\

\noindent Las oraciones interrogativas generales se utilizan para preguntar sobre la veracidad o falsedad de algo, y se responden mediante sí o no. Este tipo de preguntas se pueden hacer de dos formas:
\begin{itemize}
\item Formulando las preguntas con entonación ascendente:\\
\indent  Zerua urdina da? Bai, zerua urdina da. (¿El cielo es azul? Sí, el cielo es azul.)
\item Intercalando la partícula AL:\\
\indent Zerua urdina AL da? Bai, zerua urdina da. (¿Es el cielo azul? Si, el cielo es azul.)
\end{itemize}

\subsection{Koloreak}
\begin{itemize}
\begin{multicols}{2}
\item arrosa: rosa
\item beltza: negro
\item berdea: verde
\item gorria: rojo
\item horia: amarillo
\item laranja: naranja
\item marroia: marrón
\item morea: morado
\item urdina: azul
\item zuria: blanco
\end{multicols}
\end{itemize}

\section{A1 - Amaia Donostian bizi da}
Algunos verbos en euskera son el resultado de la unión de dos palabras: un nombre + un verbo. Éste es el caso del verbo \textbf{bizi izan}. La locución verbal bizi izan es del tipo NOR.\\
\indent Amaia Donostian \textbf{bizi da}. (Amaia vive en Donostia.)\\

\noindent La oraci\'on tiene lo que llamamos \textbf{declinaci\'on}. Del mismo que en castellano existen las preposiciones, en euskera hay posposiciones que van unidas a los nombres. Para nombrar cada caso, se utiliza el interrogativo correspondiente. Aqui estamos ante el caso NON:\\
\indent \textbf{NON} bizi da Amaia? Amaia Donostian bizi da. (D\'onde vive Amaia? Amaia vive en Donostia).\\

\noindent Para responder a la pregunta NON, a los nombres propios de lugar se les a\~nade la terminaci\'on \textbf{-(E)N}. \textbf{-EN} si acaban en constonante, \textbf{-N} si acaban en vocal.\\

\subsection{Vocabulario de Lugares}
\begin{itemize}
\begin{multicols}{2}
\item auzoa: barrio
\item herrixka: aldea, pueblo pequeño
\item herria: pueblo
\item hiria: ciudad
\item eskualdea: región
\item probintzia: provincia
\end{multicols}
\end{itemize}

\section{A1 - Amaia etxean dago}
\subsection{Egon}
\begin{itemize}
\begin{multicols}{2}
\item Ni nago. (yo estoy)
\item Hi hago. (tú estás)
\item Hura dago. (él está)\\
\item Gu gaude. (nosotros estamos)
\item Zu zaude. (tú estás)
\item Zuek zaudete. (vosotros estáis)
\item Haiek daude. (ellos están)
\end{multicols}
\end{itemize}
\indent \indent eg. Ni EZ nago Zarautzen. (Yo no estoy en Zarautz.)\\
\indent eg. Zuek EZ zaudete Bermeon. (Vosotros no estáis en Bermeo.)\\

\noindent El caso NON se utiliza principalmente para indicar sitio o lugar. Para responder al caso NON, a los \textbf{nombres comunes en singular} se le debe a\~nadir la terminaci\'on \textbf{-(e)AN}: es decir:
\begin{itemize}
\item \textbf{-N} a los que terminan en vocal \textbf{-a}.  \textit{Non daude Mikel eta Xabi? Mikel eta Xabi tabernaN daude.}
\item \textbf{-AN} a los que terminan en vocal \textbf{-e, -i, -o, -u}. \textit{Non dago Amaia? Amaia etxeAN dago.}
\item \textbf{-EAN} a los que terminan en consonante. \textit{Non zaude zu? Ni komunEAN nago.}
\end{itemize}

\noindent Los adjetivos también se declinan siguiendo las leyes de arriba. La marca se usa una sola vez, es decir, \textit{Amaia etxe berriAN dago}, o por ejemplo, \textit{Mikel eta Xabi taberna zaharrEAN daude.}\\

\noindent A los \textbf{nombres en plural y en mugagabe} hay que agregarles \textbf{-ETAN} y \textbf{-(e)TAN} respectivamente.\\
\indent \textit{Haiek gure etxeetan daude.} (Ellos están en nuestras casas.)\\
\indent \textit{Haiek zenbait etxetan daude.} (Ellos están en algunas casas.)\\

\noindent A los \textbf{nombres propios de lugar}, como ya vimos, se usa el sufijo \textbf{-(e)N}.

\subsection{Demostrativos: NON}
\begin{itemize}
\begin{multicols}{2}
\item \textbf{honetan} (en este/a)
\item \textbf{horretan} (en ese/a)
\item \textbf{hartan} (en aquel/aquella)
\item \textbf{hauetan} (en estos/as)
\item \textbf{horietan} (en esos/as)
\item \textbf{haietan} (en aquellos/as)
\end{multicols}
\end{itemize}
\indent \indent \textit{Mikel zinema horretan dago.} (Mikel está en ese cine.)

\subsection{Adverbios de Lugar: NON}
\begin{itemize}
\begin{multicols}{3}
\item \textbf{hemen} (aquí)
\item \textbf{hor} (ahí)
\item \textbf{han} (allí)
\end{multicols}
\end{itemize}
\indent \indent ej. \textit{Non dago Amaia? Amaia hemen dago.} (Amaia está aquí.)\\
\indent ej. \textit{Non dago mutila? Mutila hor dago.} (El chico está ahí.)\\
\indent ej. \textit{Non bizi da Martin? Martin han bizi da.} (Martin vive allí.)\\

\subsection{Lugares}
\begin{itemize}
\begin{multicols}{2}
\item baserria: caserío
\item denda: tienda
\item egongela: sala de estar
\item eskola: escuela
\item etxea: casa
\item kalea: calle
\item komuna: cuarto de baño
\item lantokia: lugar de trabajo
\item logela: dormitorio
\item sukaldea: cocina
\item taberna: bar
\item zinema: cine
\end{multicols}
\end{itemize}

\section{A1 - Amaia haserre dago}
\noindent En euskera se utiliza el interrogativo \textbf{NOLA} para preguntar cómo está algo o alguien.\\
\indent \textit{Nola dago Amaia? Amaia haserre dago.} (¿Cómo está Amaia? Amaia está enfadada.)\\

\noindent \textbf{ALA} es la conjunción disyuntiva que se utiliza en oraciones interrogativas.\\
\indent \textit{Nola dago Amaia, haserre ala harrituta?} (¿Cómo está Amaia, enfadada o asombrada?)\\
\indent \textit{Amaia ez dago haserre. Amaia harrituta dago.}
(Amaia no está enfadada. Amaia está asombrada.)\\

\noindent \textbf{Atención}: Aunque tengan la misma traducción en castellano, no hay que confundir las estructuras \textbf{NOLA DAGO} y \textbf{NOLAKOA DA} . Mientras que NOLA vale para preguntar sobre \textbf{situaciones pasajeras}, mediante NOLAKO se pregunta sobre \textbf{situaciones permanentes}.\\
\indent ej. \textit{Nola dago Amaia? Amaia haserre dago.} (¿Cómo está Amaia? Amaia está enfadada.)\\ \indent Ahora está enfadada, pero mañana no lo estará.\\
\indent ej. \textit{Nolakoa da Amaia? Amaia polita da.} (¿Cómo es Amaia? Amaia es bonita.)\\ \indent Ahora es bonita y mañana seguirá siendo bonita.

\subsection{Como est\'as?}
\begin{itemize}
\begin{multicols}{2}
\item beldurtuta: asustado
\item bustita: mojado
\item eserita: sentado
\item etzanda: tumbado
\item garbi: limpio
\item harrituta: asombrado
\item haserre: enfadado
\item itxita: cerrado
\item maiteminduta: enamorado
\item nekatuta: cansado
\item pozik: contento
\item puskatuta: roto
\item triste: triste
\item urduri: nervioso
\item zikin: sucio
\item zutik: de pie
\end{multicols}
\end{itemize}

\section{A1 - Noren autoa da hau?}
\noindent Utilizamos el interrogativo \textbf{NOREN} cuando queremos saber quién es el dueño de algo. En euskera, a diferencia del castellano, el caso NOREN se coloca \textit{delante} de la palabra a la que acompaña\\

\indent \textit{Noren autoa da hau?} (¿El coche de quién es éste?)\\
\indent \textit{Hau aitaren autoa da.} (Éste es el coche del padre.)\\
\indent \textit{Noren giltzak dira hauek?} (¿Las llaves de quién son éstas?) \\
\indent \textit{Hauek amonaren giltzak dira.} (Éstas son las llaves de la abuela)\\

\indent \textit{Hau EZ da aitaren autoa}. (Éste no es el coche del padre.)\\

\indent \textit{Non dago Amaia?}
(¿Dónde está Amaia?)\\
\indent \textit{Amaia aitonaren ETXEAN dago.} (Amaia está en la casa del abuelo)

\subsection{Sufijos}
\begin{itemize}
\item \textbf{—AREN} para el singular: \textit{semeAREN autoa} (el coche del hijo).
\item \textbf{—EN} para el plural: \textit{semeEN autoa} (el coche de los hijos).
\item \textbf{—(r)EN} para el indeterminado: \textit{bi semeren autoa} (el coche de dos hijos)\\
\textit{zenbait mutilen autoa} (el coche de algunos chicos)
\item \textbf{-(r)EN} para los nombres propios: \textit{Amaiaren autoa, Mikelen etxea.}
\end{itemize}

\subsection{Pronombres: NOREN}
\begin{itemize}
\begin{multicols}{2}
\item Ni: nire (mi/s)
\item Hi: hire (tu/s)
\item Hura: haren / Bere (su/s)\\
\item Gu: gure (nuestro/s,nuestra/s)
\item Zu: zure (tu/s)
\item Zuek: zuen (vuestro/s, vuestra/s)
\item Haiek: haien / euren (su/s)
\end{multicols}
\end{itemize}

\noindent \textbf{bere} y \textbf{euren} son pronombres reflexivos que indican posesión:\\
\indent \textit{Amaia haren etxean dago.} (Amaia está en su casa, en casa de alguien)\\
\indent \textit{Amaia bere etxean dago.} (Amaia está en su casa, Amaia está en casa de Amaia.)

\subsection{Familia}
\begin{itemize}
\begin{multicols}{2}
\item amona/amama: abuela
\item aitona: abuelo
\item aitona-amonak: abuelos
\item biloba: nieto/a
\item ama: madre
\item aita: padre
\item gurasoak: padres
\item semea: hijo
\item alaba: hija
\item seme-alabak: hijos
\item ahizpa: hermana (de chica)
\item arreba: hermana (de chico)
\item anaia: hermano
\item neba: hermano (de chica)
\item anai-arrebak: hermanos
\item izeba: tía
\item osaba: tío
\item osaba-izebak: tíos (tío/s y tía/s)
\item iloba: sobrino/a
\item lehengusua: primo/a
\end{multicols}
\end{itemize}

\section{A1 - Norena da auto hau?}
\noindent La diferencia con el caso NOREN es la siguiente:\\
\indent ej. \textit{Noren autoA da hau?} (¿El coche de quién es éste?)\\
\indent \textit{Hau aitaren autoA da.} (Éste es el coche del padre.)\\
\indent ej. \textit{NorenA da auto hau?} (¿De quién es este coche?)\\
\indent \textit{Auto hau aitarenA da.} (Este coche es del padre.)\\

\noindent Es decir, en el caso \textbf{NOREN} se pregunta por la pertenencia de un objeto tomando como referencia al objeto, mientras que en el caso \textbf{NORENA}, se toma como referencia a la persona.\\

\noindent Para no repetir la palabra auto dos veces, se oculta la primera de las dos, uniendo el artículo al interrogativo, en este caso es \textbf{-a}.\\

\indent \textit{NorenA da bolaluma hori? Bolaluma hori nire alabarenA da.}\\
\indent \textit{NorenAK dira liburuak? Liburuak gure semeenAK dira.}\\

\indent \textit{NorenA da zapi hau? Zapi hau nireA da.} (¿De quién es este pañuelo? Este pañuelo es mío.)\\
\indent \textit{Noren zapiA da zapi hau?} (¿El pañuelo de quién es este pañuelo?)\\
\indent \indent \textit{Zapi hau nire zapiA da.} (Este pañuelo es mi pañuelo.)

\begin{multicols}{2}
\begin{itemize}
\item NireA/AK: mío/s; mía/s
\item HireA/AK: tuyo/s; tuya/s
\item HarenA/AK / BereA/AK: suyo/s; suya/s\\
\item GureA/AK: nuestro/s; nuestra/s
\item ZureA/AK: tuyo/s; tuya/s
\item ZuenA/AK: vuestro/s; vuestra/s
\item HaienA/AK / EurenA/AK: suyo/s; suya/s
\end{itemize}
\end{multicols}
\vspace{0,2cm}
\begin{multicols}{2}
\begin{itemize}
\item Hau: Honen (a/ak) (¿de éste?)
\item Hori: Horren (a/ak) (¿de ése?)
\item Hura: Haren (a/ak) (¿de aquél?)
\item Hauek: Hauen (a/ak) (¿de éstos?)
\item Horiek: Horien (a/ak) (¿de ésos?)
\item Haiek: Haien (a/ak) (¿de aquéllos?)
\end{itemize}
\end{multicols}

\subsection{Vocabulario}
\begin{itemize}
\begin{multicols}{2}
\item sare: red
\item klera: tiza
\item baloi: balón
\item zerra: sierra
\item libragailu: desatascador
\item xurgagailu: aspirador
\item gutun: carta
\item gorbata: corbata
\item xiringa: jeringuilla
\item maleta: maletín
\end{multicols}
\end{itemize}

\section{A1 - Amaia Mikelen ondoan dago}

\subsection{Posposiciones de lugar}
\noindent Se colocan detrás de sustantivos declinados en el caso NOREN (menos los inanimados en singular). Aunque cuando un demostrativo acompaña a un nombre inanimado singular, el demostrativo sí toma el caso NOREN. Para hacer preguntas se utilizan los interrogativos \textbf{NOREN} o \textbf{ZEREN}.

\begin{itemize}
\item gainean / azpian (encima / debajo)
\item aurrean / atzean (delante / detrás)
\item ondoan (al lado)
\item barruan (dentro)
\item artean (entre)
\end{itemize}

\subsection{Ejemplos}
\subsubsection{Gainean (encima) / Azpian (debajo)}
\begin{itemize}
\begin{multicols}{2}
\item (sing) \textit{mutilAREN azpian}
\item (pl) \textit{mutilEN azpian}
\item (mgg) \textit{zenbait mutilEN azpian}\\
\item (sing) \textit{aulki gainean}
\item (pl) \textit{aulkiEN gainean}
\item (mgg) \textit{zenbait aulkiREN gainean}
\item \textit{Zure betaurrekoak mahai horren gainean daude}
\end{multicols}
\end{itemize}

\subsubsection{Aurrean (delante) / Atzean (detr\'as)}
\indent ej. \textit{Eliza udaletxe aurrean dago.} (La iglesia está delante del ayuntamiento.)\\
ej. \textit{Amaia eta Idoia mutil horien atzean daude.} (Amaia e Idoia están detrás de esos chicos.)\\
ej. \textit{Zure etxea zenbait zuhaitzen atzean dago.} (Tu casa está detrás de algunos árboles.)

\subsubsection{Ondoan (al lado)}
\indent ej. \textit{Txakurra katuaren ondoan dago.} (El perro está al lado del gato.)\\
ej. \textit{Zure etxea hondartza ondoan dago.} (Tu casa está al lado de la playa.)

\subsubsection{Barruan (dentro)}
\indent ej. \textit{Mikel auto barruan dago.} (Mikel está dentro del coche.)\\
ej. \textit{Txoriak kaiola horren barruan daude.} (Los pájaros están dentro de esa jaula.)

\subsubsection{Artean (entre)}
Esta posposición también va detrás de palabras declinadas en el caso NOREN. Pero en este caso, es con una palabra en mugagabe cuando se silencia el sufijo.\\
\indent ej. \textit{Amaia mutilen artean dago.} (sabemos qué chicos son)(Amaia está entre los chicos.)\\
\indent ej. \textit{Amaia mutil artean dago.} (no sabemos qué chicos o cuántos son)(Amaia está entre chicos.)

\subsubsection{Ejemplos de preguntas}
\textit{Noren ondoan dago Xabi?} (¿Al lado de quién está Xabi?)\\
\textit{Leireren ondoan.}
(Al lado de Leire.)\\
\textit{Xabi Leireren ondoan dago.}
(Xabi está al lado de Leire.)\\

\noindent \textit{Zeren atzean dago udaletxea?}
(¿Detrás de qué está el ayuntamiento?)\\
\textit{Banketxe atzean.}
(Detrás del banco)\\
\textit{Udaletxea banketxe atzean dago.}
(El ayuntamiento está detrás del banco.)

\subsection{Establecimientos}
\begin{itemize}
\begin{multicols}{2}
\item autobus-geltokia: estación de autobuses
\item banketxea: banco
\item eliza: iglesia
\item zerbitzugunea: área de servicio
\item hondartza: playa
\item ospitalea: hospital
\item portua: puerto
\item postetxea: oficina de correos
\item tren-geltokia: estación de trenes
\item udaletxea: ayuntamiento
\end{multicols}
\end{itemize}

\section{A1 - Amaiak liburu bat dauka}
El verbo \textbf{EDUKI} es del tipo \textbf{NORK}: necesita un sujeto y un objeto. El objeto se declina en el caso NOR. Puede ser singular, plural o indeterminado, y conforme a esto, la forma verbal cambia. El sujeto se declina en el caso NORK.\\

ej. \textit{Amaiak liburu bat dauka.} (Amaia tiene un libro.)\\
\indent \indent NORK dauka liburu bat? Amaiak. Sujeto. (¿Quién tiene un libro? Amaia.)\\
\indent \indent Zer dauka Amaiak? Liburu bat. Objeto.
(¿Qué tiene Amaia? Un libro.)\\
\indent ej. \textit{Amaiak liburuak dauzka.} (Amaia tiene libros.)\\
\indent ej. \textit{Amaiak zenbait liburu dauka.} (Amaia tiene algunos libros.)\\

\begin{itemize}
\item \textbf{-AK}: En singular. \textit{NeskAK liburu bat dauka}.
\item \textbf{-EK}: En plutal. \textit{NeskEK liburu bat daukate}.
\item \textbf{-(e)K}: En indeterminado. \textit{Zenbait NeskaK liburu bat dauka}.
\item \textbf{-(e)K}: Nombres propios. \textit{MikelEK, AmaiaK}.
\end{itemize}

\subsection{Eduki}
\begin{itemize}
\begin{multicols}{2}
\item Nik daukat / dauzkat
\item Hik daukak/n / dauzkak/n
\item Hark dauka / dauzka\\
\item Guk daukagu / dauzkagu
\item Zuk daukazu / dauzkazu
\item Zuek daukazue / dauzkazue
\item Haiek daukate / dauzkate
\end{multicols}
\end{itemize}

\subsection{N\'umeros}
\begin{itemize}
\begin{multicols}{2}
\item 0: zero
\item 1: bat
\item 2: bi
\item 3: hiru
\item 4: lau
\item 5: bost
\item 6: sei
\item 7: zazpi
\item 8: zortzi
\item 9: bederatzi
\item 10: hamar
\end{multicols}
\end{itemize}

\noindent Los números en euskera, con la excepción de bat, se colocan \textbf{delante} del sustantivo.\\
\indent ej. \textit{Amaiak liburu bat dauka}.\\
\indent ej. \textit{Amaiak bi alaba polit dauzka}.\\

\noindent Si un sustantivo tiene un número delante, puede acompañarse o no del \textbf{artículo}:\\
\indent ej. \textit{Amaiak BOST liburu dauzka} (Los libros son desconocidos, Amaia tiene CINCO libros).\\
\indent ej. \textit{Amaiak BOST liburuAK dauzka.} (Los libros son conocidos, Amaia tiene los cinco libros.)\\

\noindent \textbf{BAT} (uno / una) y \textbf{BATZUK} (unos / unas) hacen la función de \textbf{artículos indefinidos}.

\begin{multicols}{2}
\noindent \textit{Hau etxeA da}. (Ésta es la casa.)\\
\textit{Hau etxe BAT da}. (Ésto es una casa.)\\
\textit{Hauek etxeAK dira}. (Éstas son las casas.)\\
\textit{Hauek etxe BATZUK dira}. (Éstos son unas casas.)
\end{multicols}

\noindent \textbf{BATEK / BATZUEK} son las formas que toma el artículo indeterminado en el caso \textbf{NORK}.\\
\indent \textit{Emakume BATEK lau etxe dauzka Iruñean}.
(Una mujer tiene cuatro casas en Pamplona.)\\
\indent \textit{Emakume BATZUEK lau etxe dauzkate Iruñean}.
(Unas mujeres tienen cuatro casas en Pamplona.)

\section{A1 - Neska horrek ez dauka libururik}
\subsection{Demostrativos: NORK}
\begin{itemize}
\begin{multicols}{2}
\item honek (este, esta)
\item horrek (ese, esa)
\item hark (aquel, aquella)
\item hauek (estos, estas)
\item horiek (esos, esas)
\item haiek (aquellos, aquellas)
\end{multicols}
\end{itemize}

\textit{Neska honek ez dauka liburua.} (Esta chica no tiene el libro.)\\
\indent \textit{Neska honek ez dauka liburuRIK.} (Esta chica no tiene libros.)\\
\noindent En el primer ejemplo se da a entender que la chica no tiene un libro concreto. La segunda oración implica que la chica no tiene ningún libro. Al caso utilizado en la segunda oración se le denomina \textbf{PARTITIVO}. Con el partitivo el verbo siempre va en singular.\\

\noindent \textbf{Asko}, \textbf{gutxi} y \textbf{pilo bat} son numerales indeterminados. \textit{Asko} y \textit{pilo bat} implican abundancia, mientras que \textit{gutxi} implica escasez. Los tres suelen colocarse detrás del sustantivo. \textit{Asko} y \textit{gutxi} se declinan en indeterminado o mugagabe. Cuando los objetos se toman como grupo, o cuando se trata de nombres incontables (como gente/jende, dinero/diru), el verbo suele ir en singular, aunque podr\'ia ser en plural.   Con el numeral indeterminado pilo bat, el verbo suele ir en singular.\\

\noindent Se utiliza \textbf{ZENBAT…?} para preguntar sobre cantidades determinadas o indeterminadas.\\
\indent \textit{Zenbat anaia dauzka Josebak? Josebak bi anaia dauzka.}\\
\indent \textit{Zenbat diru dauka Anderrek? Anderrek diru asko dauka.}\\
\noindent Con el interrogativo zenbat el verbo puede ir en singular o plural. Con sustantivos incontables el verbo se conjuga en singular.

\subsection{Eragiketa matematikoak}
\begin{multicols}{2}
\begin{itemize}
\item batuketa: suma
\item kenketa: resta
\item zatiketa: división
\item biderketa: multiplicación
\item bider: por
\item gehi: más
\item ken: menos
\item zati: entre
\item berdin: igual a
\item zenbat da…?: ¿cuánto es?
\end{itemize}
\end{multicols}

\section{A1 - Amaia dendara doa}
\subsection{Joan}
\noindent JOAN es un verbo del tipo NOR.
\begin{multicols}{2}
\begin{itemize}
\item Ni $>$ noa (voy)
\item Hi $>$ hoa (vas)
\item Hura $>$ doa (va)\\
\item Gu $>$ goaz (vamos)
\item Zu $>$ zoaz (vas)
\item Zuek $>$ zoazte (vais)
\item Haiek $>$ doaz (van)
\end{itemize}
\end{multicols}

\indent \textit{Amaia dendaRA doa.} (Amaia va a la tienda.)\\
\indent \textit{Xabi eta Idoia lantokiRA doaz.} (Xabi e Idoia van al lugar de trabajo.)\\

\noindent Los lugares se declinan en el caso NORA:
\begin{itemize}
\item \textbf{-(e)RA}: singular.\\
\indent \textit{Zu komunERA zoaz.} (Tú vas al cuarto de baño.)
\item \textbf{-ETARA}: plural.\\
\indent \textit{Ni baserriETARA noa.}(Yo voy a los caseríos.)
\item \textbf{-(e)TARA}: indeterminado.\\
\indent \textit{Ni zenbait baserriTARA noa.} (Yo voy a algunos caseríos.)
\item \textbf{-(e)RA}: Nombres Propios.\\ 
\indent \textit{Ni ZarautzERA noa.} (Yo voy a Zarautz.)
\end{itemize}

\noindent Las preguntas se hacen mediante el interrogativo NORA.\\
\indent \textit{Nora zoaz zu? Ni amaren etxera noa.} (¿A dónde vas? Voy a casa de mi madre.)\\

\noindent En las oraciones negativas cambia el orden de los elementos.\\
\indent \textit{Amaia etxera al doa?} (¿Amaia va a casa?)\\
\indent \textit{Ez, Amaia ez doa etxera, Amaia tabernara doa.} (No, Amaia no va a casa, Amaia va al bar.)

\subsection{Adverbios de Lugar: NORA}
\begin{itemize}
\item hemen $>$ hona (aquí, a este lugar)
\item hor $>$ horra (ahí, a ese lugar)
\item han $>$ hara (allí, a aquel lugar)
\end{itemize}
\indent \textit{Mikelen ama horra doa.} (La madre de Mikel va ahí, a ese lugar.)

\subsection{Demostrativos: NORA}
\begin{multicols}{2}
\begin{itemize}
\item hau $>$ honetara (a esta, a este)
\item hori $>$ horretara (a esa, a ese)
\item hura $>$ hartara (a aquella, a aquel)
\item hauek $>$ hauetara (a estas, a estos)
\item horiek $>$ horietara (a esas, a esos)
\item haiek $>$ haietara (a aquellas, a aquellos)
\end{itemize}
\end{multicols}
\indent \textit{Haiek taberna hartara doaz.} (Ellos van a aquel bar.)\\

\section{A1 - Amaia amonarengana joan da}
\noindent El caso NORA se utiliza con nombres inanimados, mientras que \textbf{NORENGANA} con animados.\\
\indent \textit{Mikel eta Xabi Idoiarengana joan dira.} (Mikel y Xabi han ido donde Idoia.)\\
\begin{itemize}
\item \textbf{-ARENGANA}: singular:\\ \textit{Mikel emakumearengana joan da.} (Mikel ha ido donde la mujer.)
\item \textbf{-ENGANA}: plural:\\
\textit{Mikel emakumeengana joan da.}
(Mikel ha ido donde las mujeres.)
\item \textbf{-(r)ENGANA}: mugagabe\\
\textit{Mikel bi emakumerengana joan da.} (Mikel ha ido donde dos mujeres.)
\end{itemize}

\indent \textit{Norengana joan da Amaia?} (¿Donde quién ha ido Amaia?)\\
\indent \textit{Amaia ez da bere mutil-lagunarengana joan.} (Amaia no ha ido donde su novio.)

\subsection{Demostrativos: NORENGANA}
\begin{multicols}{2}
\begin{itemize}
\item hau $>$ honengana
\item hori $>$ horrengana
\item hura $>$ harengana
\item hauek $>$ hauengana
\item horiek $>$ horiengana
\item haiek $>$ haiengana
\end{itemize}
\end{multicols}
\indent \textit{Amaia emakume horrengana joan da.} (Amaia ha ido donde esa mujer.)

\subsection{Pronombres: Norengana}
\begin{multicols}{2}
\begin{itemize}
\item (ni) $>$ niregana
\item (hi) $>$ hiregana
\item (hura) $>$ harengana / beregana\\
\item (gu) $>$ guregana
\item (zu) $>$ zuregana
\item (zuek) $>$ zuengana
\item (haiek) $>$ haiengana / eurengana
\end{itemize}
\end{multicols}
\indent \textit{Amaia zuregana joan da.} (Amaia ha ido donde ti.)

\subsection{JOAN + IZAN}
\indent \indent \textit{Amaia amonarengana doa.}
(Amaia va donde la abuela.)\\
\indent \textit{Amaia amonarengana joan da.}
(Amaia ha ido donde la abuela.)\\
\noindent La primera oración da a entender que Amaia está yendo donde la abuela en este momento, la segunda comunica que Amaia ya ha llegado a casa de la abuela. El verbo JOAN es del tipo NOR. Para indicar una acción ya terminada, se unen el participio de JOAN (joan) y las formas verbales de IZAN.\\

\subsection{Expresiones temporales}
\begin{multicols}{2}
\begin{itemize}
\item goiza: mañana
\item goizean: por la mañana
\item eguerdia: mediodía
\item eguerdian: al mediodía
\item arratsaldea: tarde
\item arratsaldean: por la tarde
\item gaua: noche
\item gauean: por la noche
\item gauerdia: medianoche
\item gauerdian: a medianoche
\item duela gutxi: hace poco
\end{itemize}
\end{multicols}
\noindent Para responder a la pregunta NOIZ, los nombres se declinan en el caso NON:\\
\indent \textit{Noiz joan da Amaia sendagilearengana?} (¿Cuándo ha ido Amaia donde el médico?)\\
\indent \textit{Amaia gaur goizean joan da sendagilearengana.} (Amaia ha ido esta mañana donde el médico.)
\noindent \textbf{Nota}: NO son correctas expresiones como goiz honetan (esta mañana), arratsalde honetan (esta tarde) y gau honetan (esta noche). En vez de éstas han de utilizarse estas otras: gaur goizean, gaur arratsaldean o gaur gauean.

\section{A1 - Amaia amonaren etxetik dator}
\subsection{Etorri}
\begin{multicols}{2}
\begin{itemize}
\item ni nator (yo vengo)
\item hi hator (tú vienes)
\item hura dator (él viene)\\
\item gu gatoz (nosotros venimos)
\item zu zatoz (tú vienes)
\item zuek zatozte (vosotros venís)
\item haiek datoz (ellos vienen)
\end{itemize}
\end{multicols}
\indent \textit{Amaia amonaren etxetik dator.} (Amaia viene de la casa de la abuela.)\\
\indent \textit{Amaia amonaren etxetik etorri da.} (Amaia ha venido de la casa de la abuela.)

\subsection{Sufijos}
En estas oraciones se utilizan los sufijos correspondientes al caso \textbf{NONDIK}:
\begin{itemize}
\item \textbf{-(e)TIK}: singular:\\
\textit{Mikel okindegiTIK dator.}
(Mikel viene de la panadería.)\\
\textit{Zure laguna komunETIK dator.}
(Tu amigo viene del cuarto de baño.)
\item \textbf{-ETATIK}: plural:\\
\textit{Mikel arropa dendETATIK dator.}
(Mikel viene de las tiendas de ropa.)
\item \textbf{-(e)TATIK}: indeterminado:\\
\textit{Mutil horiek hainbat fruta-dendaTATIK datoz.} (Esos chicos vienen de varias fruterías.)\\
\textit{Zure lagunak zenbait komunETATIK datoz.} (Tus amigos vienen de algunos cuartos de baño.)
\end{itemize}
A los nombres propios se les agrega
\textbf{-(e)TIK}: \\
\indent Zu DonostiaTIK zatoz. (Tú vienes de Donostia.)\\
\indent Zuek BilboTIK zatozte. (Vosotros venís de Bilbao.)\\
Pero si terminan en consonante pueden tomar dos formas, \textbf{-TIK} o \textbf{-ETIK}.\\
\indent \textit{Ni Paristik nator.} (Yo vengo de París.)\\
\indent \textit{Ni Parisetik nator. }(Yo vengo de París.)\\
Pero si terminan en \textbf{-N} o \textbf{-L}, se les agrega \textbf{-DIK} o \textbf{-ETIK}, no \textbf{-TIK}.\\
\indent \textit{Irundik etorri zara.} (Has venido de Irún.)\\
\indent \textit{Irunetik etorri zara.} (Has venido de Irún.)

\subsection{Demostrativos: NONDIK}
\begin{multicols}{2}
\begin{itemize}
\item hau $>$ honetatik (de esta, este)
\item hori $>$ horretatik (de esa, ese)
\item hura $>$ hartatik (de aquel, aquella)
\item hauek $>$ hauetatik (de estas, estos)
\item horiek $>$ horietatik (de esas, esos)
\item haiek $>$ haietatik (de aquellos, aquellas)
\end{itemize}
\end{multicols}
\textit{Zure ama supermerkatu horietatik dator.} (Tu madre viene de esos supermercados.)

\subsection{Adverbios de Lugar: NONDIK}
\begin{itemize}
\item hemen $>$ hemendik (de aquí)
\item hor $>$ hortik (de ahí)
\item han $>$ handik (de allí)
\end{itemize}
\indent \indent \textit{Mikel hortik dator, liburu-dendatik.} (Mikel viene de ahí, de la tienda de libros.)

\subsection{Establecimientos}
\begin{multicols}{2}
\begin{itemize}
\item arrandegia: pescadería
\item arropa-denda: tienda de ropa
\item barazki-denda: verdulería
\item bidaia-agentzia: agencia de viajes
\item harategia: carnicería
\item fruta-denda: frutería
\item izozki-denda: heladería
\item liburu-denda: librería
\item okindegia: panadería
\item supermerkatua: supermercado
\item tabako-denda: estanco
\end{itemize}
\end{multicols}
\indent \textit{Ez, Amaia ez da fruta-dendatik etorri, Amaia okindegitik etorri da.}\\
\indent (No, Amaia no ha venido de la frutería, Amaia ha venido de la panadería.)

\section{A1 - Amaiak sagar bat dakar bere amonarentzat}
\textbf{EKARRI} es un verbo del tipo NORK. Se traduce como traer.

\subsection{Ekarri}
\begin{itemize}
\item Nik dakart (sing) / dakartzat (pl)
\item Hik dakark/n (sing) / dakartzak/n (pl)
\item Hark dakar (sing) / dakartza (pl)
\item Guk dakargu (sing) / dakartzagu (pl)
\item Zuk dakarzu (sing) / dakartzazu (pl)
\item Zuek dakarzue (sing) / dakartzazue (pl)
\item Haiek dakarte (sing) / dakartzate (pl)
\end{itemize}

\indent \textit{Zer dakarzu zuk platerean?} (¿Qué traes tú en el plato?)\\
\indent \textit{Nik lau banana dakartzat platerean.} (Yo traigo cuatro plátanos en el plato.)\\

\noindent Para convertir las oraciones vistas arriba en negativas, debemos utilizar el PARTITIVO:\\
\indent \textit{Nik ez dakart bananarik platerean.} (Yo no traigo plátanos en el plato.)\\

\noindent El objeto puede ser singular, plural o indeterminado. Si el objeto es indeterminado, el verbo puede ir en singular o plural. En cambio, con el numeral indeterminado pilo bat, cuando los objetos son tomados como un grupo, o con nombres incontables, se utilizan las formas verbales singulares.\\
\indent \textit{Nik udare pilo bat dakart hemen.} (Yo traigo un montón de peras aquí.)\\
\indent \textit{Amaiak diru asko dakar sakelan.} (Amaia trae un montón de dinero en el bolsillo.)

\subsection{Norentzat}
\indent \indent \textit{Nik udare pilo bat dakart irakaslearentzat.} (Yo traigo un montón de peras para la profesora.)\\
\begin{itemize}
\item \textbf{-ARENTZAT}: singular:\\
\textit{Opari hau neskARENTZAT da.}
(Este regalo es para la chica.)
\item \textbf{-ENTZAT}: plural:\\
\textit{Opari hau mutilENTZAT da.}
(Este regalo es para los chicos.)
\item \textbf{-(r)ENTZAT}: indeterminado:\\
\textit{Opari hau zenbait mutilENTZAT da.}
(Este regalo es para algunos chicos.)\\
\textit{Opari hau zenbait neskaRENTZAT da.}
(Este regalo es para algunas chicas.)
\end{itemize}

\subsection{Pronombres: NORENTZAT}
\begin{itemize}
\begin{multicols}{2}
\item niretzat
\item hiretzat
\item harentzat / beretzat\\
\item guretzat
\item zuretzat
\item zuentzat
\item haientzat / eurentzat
\end{multicols}
\end{itemize}

\subsection{Demostrativos: NORENTZAT}
\begin{itemize}
\begin{multicols}{2}
\item honentzat
\item horrentzat
\item harentzat
\item hauentzat
\item horientzat
\item haientzat
\end{multicols}
\end{itemize}

\subsection{Adjetivos}
\indent \indent \textit{Amaiak sagar txiki-txiki bat dakar.}
(Amaia trae una manzana muy pequeña.)\\
\noindent Para subrayar la pequeñez de la manzana de Amaia se repite el adjetivo. Las dos palabras se unen mediante un guión. La primera palabra no lleva ningún sufijo, pero sí puede llevarlos la segunda.\\

\noindent También puede utilizarse con adverbios, los cuales no toman ningún sufijo:\\
\indent Merkatura poliki-poliki doa.
(Va muy despacio al mercado.)

\subsection{Frutas}
\begin{itemize}
\begin{multicols}{2}
\item arana: ciruela
\item banana: plátano
\item gerezia: cereza
\item laranja: naranja
\item limoia: limón
\item mandarina: mandarina
\item marrubia: fresa
\item masusta: mora
\item mertxika: melocotón
\item sagarra: manzana
\item udarea: pera
\end{multicols}
\end{itemize}

\section{A1 - Amaia sagar bat darama poltsan}
\noindent \textbf{ERAMAN} es un verbo del tipo NORK. Se puede traducir como llevar.

\subsection{Eraman}
\begin{itemize}
\item Nik daramat (sing) / daramatzat (pl)
\item Hik daramak/n (sing) / daramatzak/n (pl)
\item Hark darama (sing) / daramatza (pl)
\item Guk daramagu (sing) / daramatzagu (pl)
\item Zuk daramazu (sing) / daramatzazu (pl)
\item Zuek daramazue (sing) / daramatzazue (pl)
\item Haiek daramate (sing) / daramatzate (pl)
\end{itemize}
\indent \indent \textit{Nire amonak eskumuturreko bat darama eskumuturrean.} (Mi abuela lleva una pulsera en la muñeca)\\
\indent \textit{Zuek zenbait sagar daramatzazue poltsan.} (Vosotros lleváis algunas manzanas en la bolsa.)\\

\noindent Estas formas indican que la acción que señala el verbo ocurre en el mismo momento en el que se dice la oración. Para dar a entender que la acción está terminada, deberemos juntar el participio del verbo ERAMAN o EKARRI con las formas verbales sintéticas del verbo UKAN.
\begin{itemize}
\item Nik eraman / ekarri dut $>$ ditut
\item Hik eraman / ekarri duk/n $>$ dituk/n
\item Hark eraman / ekarri du $>$ ditu
\item Guk eraman / ekarri dugu $>$ ditugu
\item Zuk eraman / ekarri duzu $>$ dituzu
\item Zuek eraman / ekarri duzue $>$ dituzue
\item Haiek eraman / ekarri dute $>$ dituzte
\end{itemize}
\indent \indent \textit{Nire lagunak liburu bat dakar orain.} (Mi amigo trae un libro ahora.)\\
\indent \textit{Nire lagunak liburu bat ekarri du gaur goizean.} (Mi amigo ha traido un libro esta mañana.)\\
\indent \textit{Nire lagunak \textbf{EZ} du libururik ekarri gaur goizean.}\\
\indent \textit{Zuek bi telebista daramatzazue orain.}
(Vosotros lleváis dos televisores ahora.)\\
\indent \textit{Bi telebista eraman dituzue gaur arratsaldean.} (Habéis llevado dos televisores esta tarde.)\\
\indent \textit{Zuek \textbf{EZ} duzue telebistarik eraman gaur arratsaldean.}

\subsection{Ordua}
Para preguntar por la hora, se utiliza la pregunta \textit{Zer ordu da?}
\begin{multicols}{2}
\begin{itemize}
\item 1:00 $>$ ordu bata
\item 1:05 $>$ ordu bata eta bost
\item 2:00 $>$ ordu biak
\item 2:10 $>$ ordu biak eta hamar
\item 3:00 $>$ hirurak
\item 3:15 $>$ hiru eta laurdenAK o,\\
\indent\indent\indent HirurAK eta laurden
\item 4:00 $>$ laurak
\item 4:20 $>$ laurak eta hogei
\item 5:00 $>$ bostak
\item 5:25 $>$ bostak eta hogeita bost
\item 6:00 $>$ seiak
\item 6:30 $>$ sei eta erdiAK
\item 6:35 $>$ zazpiak hogeita bost gutxi
\item 7:00 $>$ zazpiak
\item 7:40 $>$ zortziak hogei gutxi
\item 8:00 $>$ zortziak
\item 8:45 $>$ bederatziak laurden gutxi
\item 9:00 $>$ bederatziak
\item 9:50 $>$ hamarrak hamar gutxi
\item 10:00 $>$ hamarrak
\item 10:55 $>$ hamaikak bost gutxi
\item 11:00 $>$ hamaikak
\item 12:00 $>$ hamabiak
\end{itemize}
\end{multicols}
\noindent Pero para responder a la pregunta \textit{Noiz?/Zer ordutan..?} se inclina en el caso NON.
\begin{multicols}{2}
\begin{itemize}
\item :00 -ETAN $>$ adb. ordu bietan
\item :05 -ean $>$ adb. ordu biak eta bostean
\item :10 -ean $>$ adb. ordu biak eta hamarrean
\item :15 -ETAN $>$ adb. ordu bi eta laurdenetan o, 
\indent\indent\indent ordu biak eta laurdenetan
\item :20 -an $>$ adb. ordu biak eta hogeian
\item :25 -ean $>$ adb. ordu biak eta hogeita bostean
\item :30 -ETAN $>$ adb. ordu bi eta erdietan
\item :35 -TAN $>$ adb. hirurak hogeita bost gutxitan
\item :40 -TAN $>$ adb. hirurak hogei gutxitan
\item :45 -TAN $>$ adb. hirurak laurden gutxitan
\item :50 -TAN $>$ adb. hirurak hamar gutxitan
\item :55 -TAN $>$ adb. hirurak bost gutxitan
\end{itemize}
\end{multicols}

\noindent O bien se utilizan las siguientes expresiones:


\indent \textit{8:30 Goizeko zortzi eta erdiak dira.} (Son las ocho y media de la mañana.)\\
\indent \textit{12:10 Eguerdiko hamabiak eta hamar dira.} (Son las doce y diez del mediodía.)\\
\indent \textit{18:45 Arratsaldeko zazpiak laurden gutxi dira.} (Son las siete menos cuarto de la tarde.)\\
\indent \textit{21:40 Gaueko hamarrak hogei gutxi dira.} (Son las diez menos veinte de la noche.)\\

\section{A1 - Bai auto garestia!}
\noindent \textbf{BALIO IZAN} es un verbo formado por el nombre balio y el verbo izan. Balio izan es un verbo del tipo NORK, y por lo tanto, se acompaña de formas verbales de ukan. Se traduce como \textit{valer}.\\
\noindent \textbf{NAHI IZAN} también es NORK y se acompaña de formas verbales de ukan. Se traduce como \textit{querer}.\\
\indent \textit{Liburu honek hamabi euro balio du.} (Este libro vale doce euros.)\\
\indent \textit{Mutil haiek zapata horiek nahi dituzte. }(Aquellos chicos quieren esos zapatos.)\\

\noindent Para preguntar por el precio de algo se utilizan las siguientes construcciones:
\begin{itemize}
\item \textit{Zenbat balio du...?} (¿Cuánto vale...?)
\item \textit{Zenbat da...?} (¿Cuánto es...?)
\item \textit{Zenbatean dago...?} (¿A cuánto está...?)
\end{itemize}

\indent \textit{\textbf{Zenbat balio du} auto horrek? Auto horrek bi mila euro \textbf{balio du}.}\\\indent(¿Cuánto vale ese coche? Ese coche vale dos mil euros.)\\
\indent \textit{\textbf{Zenbat da} afaria? Afaria laurogeita hamar euro \textbf{da}.}\\\indent(¿Cuánto es la cena? La cena son noventa euros.)\\
\indent \textit{Zenbat da? Lau eta hamar.}\\\indent(¿Cuánto es? Cuatro con diez.)\\
\indent \textit{\textbf{Zenbatean dago} sagar kiloa? Sagar kiloa euro bat eta laurogei zentimon \textbf{dago}.}\\\indent(¿A cuánto está el kilo de manzanas? El kilo de manzanas está a un euro con ochenta céntimos.)\\

\indent \textit{Bai auto \textbf{garestia}!} (¡Qué coche tan \textbf{caro}!)\\
\indent \textit{Produktu hauek \textbf{merke-merke} daude!} (¡Estos productos están \textbf{baratísimos}!)\\
\indent \textit{Galtza hauek oso garestiak dira, horiek, ordea, merkeak.}\\\indent(Estos pantalones son muy caros, ésos, en cambio, baratos.)\\
\indent \textit{Hori \textbf{mauka} hori!} (¡Menuda ganga!)\\


\indent \textit{\textbf{Kilo erdi} sagar nahi dut.}
(Quiero medio kilo de manzanas.)\\
\indent \textit{\textbf{Kilo bat} udare dakart poltsan.}
(Traigo un kilo de peras en la bolsa.)\\
\indent \textit{\textbf{Kilo eta erdi} gerezi nahi dut, mesedez.}
(Quiero kilo y medio de cerezas, por favor.)\\
\indent \textit{\textbf{Bi kilo} patata daukat hemen.}
(Tengo dos kilos de patatas aquí.)\\
\indent \textit{\textbf{Berrehun gramo} txorizo eta ehun gramo urdaiazpiko, mesedez.}\\
\indent (Doscientos gramos de chorizo y cien gramos de jamón, por favor.)

\begin{itemize}
\begin{multicols}{2}
\item \textit{dozena erdi} (media docena)
\item \textit{dozena bat} (una docena)
\item \textit{dozena eta erdi} (docena y media)
\item \textit{bi dozena} (dos docenas)
\item \textit{litro erdi} (medio litro)
\item \textit{litro bat} (un litro)
\item \textit{litro eta erdi} (litro y medio)
\item \textit{bi litro} (dos litros)
\end{multicols}
\end{itemize}

\noindent La construcci\'on (envase + \textbf{contenido}) hace referencia al contenido. \\ \noindent La construcción (contenido + \textbf{envase}), en cambio, hace referencia al envase.\\
\indent \textit{Hiru zaku patata ekarri zituen aitonak baserritik.} (El abuelo trajo tres sacos de patatas del caserío.)
\indent \textit{Amonak patata-zaku batean ekarri zituen sagarrak.} (La abuela trajo las manzanas en un saco de patatas.)

\subsection{Comidas y bebidas}
\begin{multicols}{2}
\begin{itemize}
\item ardo: vino
\item arrain: pescado
\item arrautza: huevo
\item azukre: azúcar
\item esne: leche
\item gazta: queso
\item haragi: carne
\item ogi: pan
\item olio: aceite
\item patata: patata
\item txorizo: chorizo
\item ur: agua
\item uraza: lechuga
\item urdaiazpiko: jamón
\item zuku: zumo
\end{itemize}
\end{multicols}

\section{A2 - Amaia pastel bat egiten ari da}
\noindent Con los verbos EGIN, IRTEN, SARTU, EDAN, JAN e IKUSI y para indicar que la acción está ocurriendo en el momento de decir la oración, se emplea el verbo \textbf{ARI IZAN}. ARI IZAN es del tipo NOR. Por lo tanto, todos los verbos utilizados con él toman las formas verbales de IZAN como auxiliares.\\

\indent \textit{Amaia pastel bat \textbf{egiten ari da}.}
(Amaia está haciendo un pastel.)\\
\indent \textit{Mikel etxetik \textbf{irteten ari da}.}
(Mikel está saliendo de la casa.)\\
\indent \textit{Emakume hori auto horretan \textbf{sartzen ari da}.}
(Esa mujer está entrando en ese coche.)\\
\indent \textit{Idoia baso bat ardo \textbf{edaten ari da}.}
(Idoia está bebiendo un vaso de vino.)\\
\indent \textit{Mutil hura sagar bat \textbf{jaten ari da}.}
(Aquel chico está comiendo una manzana.)\\
\indent \textit{Ni telebista \textbf{ikusten ari naiz}.}
(Yo estoy viendo la televisión.)\\

\indent \textit{Amaia \textbf{EZ} da pastel bat egiten ari.}
(Amaia no está haciendo un pastel.)\\
\indent \textit{Ni EZ naiz telebista ikusten ari.}
(Yo no estoy viendo la televisión.)\\

\noindent Cuando se quiere subrayar un elemento de la oración este elemento se coloca delante de ARI y el verbo subordinado pasa al final.\\
\indent \textit{Amaia pastel bat \textbf{egiten} ARI da.}\\
\indent \textit{Amaia \textbf{pastel bat} ARI da egiten.}\\
\indent \textit{\textbf{Amaia} ARI da pastel bat egiten.}\\

\subsection{Eguraldia}
\noindent Prozesuak (procesos):
\begin{multicols}{2}
\begin{itemize}
\item haizea egin (hacer viento)
\item zirimiria egin (hacer sirimiri)
\item euria egin (llover)
\item kazkabarra egin (granizar)
\item elurra egin (nevar)
\end{itemize}
\end{multicols}

\noindent Egoerak (situaciones):
\begin{multicols}{2}
\begin{itemize}
\item eguraldi ona egin (hacer buen tiempo)
\item eguraldi txarra egin (hacer mal tiempo)
\item hotz egin (hacer frío)
\item bero egin (hacer calor)
\item fresko egin (hacer fresco)
\item sargori egin (hacer bochorno)
\end{itemize}
\end{multicols}

\noindent RECORDAR: \\
\textbf{Mañana} procesos: egingo du / situaciones: egingo du\\
\textbf{Hoy} procesos: egin du / situaciones: egin du\\
\textbf{Ahora} procesos: ari du / situaciones: egiten du\\


\indent \textit{\textbf{Bihar} euria \textbf{egingo du}.}
(Mañana lloverá.)\\
\indent \textit{\textbf{Bihar} hotz \textbf{egingo du}.}
(Mañana hará frío.)\\
\indent \textit{\textbf{Gaur} euria \textbf{egin du}.}
(Hoy ha llovido.)\\
\indent \textit{\textbf{Gaur} hotz \textbf{egin du}.}
(Hoy ha hecho frío.)\\
\indent \textit{\textbf{Orain} euria \textbf{ari du}.}
(Ahora llueve.)\\
\indent \textit{\textbf{Orain} hotz \textbf{egiten du}.}
(Ahora hace frío.)\\

\section{A2 - Amaia mendira joan zen larunbatean}
Las formas verbales de las acciones que empezaron y terminaron en el pasado se crean uniendo al participio del verbo principal las formas verbales en pasado de UKAN o IZAN.\\
\indent \textit{Zu amonaren etxera joan zinen astelehenean.}
(Tú fuiste a casa de la abuela el lunes.)\\
\indent \textit{Nik pastel bat egin nuen atzo zuretzat.}
(Yo hice un pastel ayer para tí.)\\

\subsection{NOR: etorri, joan, irten, sartu, egon}
\begin{itemize}
\begin{multicols}{2}
\item Ni $>$ nintzen
\item Hi $>$ hintzen
\item Hura $>$ zen\\
\item Gu $>$ ginen
\item Zu $>$ zinen
\item Zuek $>$ zineten
\item Haiek $>$ ziren
\end{multicols}
\end{itemize}

\subsection{NOR-NORK: eduki/izan, eraman, ekarri, egin, edan, jan, ikusi}
\begin{itemize}
\begin{multicols}{2}
\item Nik $>$ nuen / nituen
\item Hik $>$ huen / hituen
\item Hark $>$ zuen / zituen\\
\item Guk $>$ genuen / genituen
\item Zuk $>$ zenuen / zenituen
\item Zuek $>$ zenuten / zenituzten
\item Haiek $>$ zuten / zituzten
\end{multicols}
\end{itemize}

\noindent \textbf{Atención:} EDUKI e IZAN se pueden utilizar como verbos principales en el mismo contexto:\\
\indent \textit{Arazo asko eduki zituen Mikelek eskolan.}\\
\indent \textit{Arazo asko izan zituen Mikelek eskolan.}\\
\indent (Mikel tuvo muchos problemas en la escuela.)\\



\indent \textit{Zu amonaren etxera \textbf{joan zara} gaur goizean.}
(Tú has ido a casa de la abuela esta mañana.)\\
\indent \textit{Zu amonaren etxera \textbf{joan zinen} astelehenean.}
(Tú fuiste a casa de la abuela el lunes.)\\

\indent \textit{Nik pastel bat \textbf{egin dut} gaur eguerdian zuretzat.}
(Yo he hecho un pastel este mediodía para tí.)\\
\indent \textit{Nik pastel bat \textbf{egin nuen} atzo zuretzat.}
(Yo hice un pastel ayer para tí.)\\

\subsection{Egunen izenak}
\begin{itemize}
\begin{multicols}{2}
\item astelehena: lunes
\item asteartea: martes
\item asteazkena: miércoles
\item osteguna: jueves
\item ostirala: viernes
\item larunbata: sábado
\item igandea: domingo
\end{multicols}
\end{itemize}

\noindent NOIZ galderari erantzuteko asteko egunen izenak NON.\\
\indent \textit{Ni Mikelen etxera joan nintzen astelehenean.}
(Yo fui a casa de Mikel el lunes.)\\
\indent \textit{Ainhoa Gorkaren herrian egon zen igandean.}
(Ainhoa estuvo en el pueblo de Gorka el domingo.)\\
\indent \textit{Ni Mikelen etxera joan nintzen astelehen goizean.}\\
\indent \textit{Ainhoa Gorkaren herrian egon zen igande eguerdian.}\\
\indent \textit{Zuek lagun hori ikusi zenuten ostegun gauean.}\\

\begin{itemize}
\begin{multicols}{2}
\item atzo: ayer
\item herenegun: antes de ayer
\item asteburuan: durante el fin de semana
\item joan den astean: la semana pasada
\item joan den hilabetean: el mes pasado
\item joan den urtean = iaz: el año pasado
\item udaberrian: en primavera
\item udan: en verano
\item udazkenean: en otoño
\item neguan: en invierno
\end{multicols}
\end{itemize}

\section{A2 - Amaia bihar joango da Bilbora Mikelekin}
\noindent El sufijo \textbf{-go}/\textbf{-ko} vale para crear formas verbales del futuro. -go se utiliza con verbos terminados en -N, y -ko con el resto.\\
\indent \textit{Mikel datorren astean Zarautzen egongo da.}
(Mikel la semana que viene estará en Zarautz.)\\
\indent \textit{Nik bihar liburu hori ekarriko dut etxetik.}
(Yo mañana traeré ese libro de casa.)\\
\indent \textit{Zuek etzi zuen lagunen futbol partida ikusiko duzue.}\\
\indent (Vosotros pasado mañana veréis el partido de fútbol de vuestros amigos.)\\

\noindent El caso \textbf{NOREKIN} (NOREN + KIN) equivale a la preposici\'on \textit{con}.
\begin{itemize}
\item \textbf{-AREKIN}: singular:\\
\textit{Amaia bere senarrAREKIN joango da Bilbora.} (Amaia irá a Bilbo con su marido.)
\item \textbf{-EKIN}: plural:\\
\textit{Amaia bere lagunEKIN etorriko da festara.} (Amaia vendrá a la fiesta con sus amigos.)
\item \textbf{-(r)EKIN}: indeterminado:\\
\textit{Amaia zenbait bizilagunEKIN joango da udaletxera.}\\
(Amaia irá al ayuntamiento con algunos vecinos.)\\
\textit{Amaia lau ikasleREKIN etorriko da etxera.}
(Amaia vendrá a casa con cuatro estudiantes.)
\end{itemize}

\subsection{Demostrativos: NOREKIN}
\begin{itemize}
\begin{multicols}{2}
\item (hau) $>$ honekin
\item (hori) $>$ horrekin
\item (hura) $>$ harekin
\item (hauek) $>$ hauekin
\item (horiek) $>$ horiekin
\item (haiek) $>$ haiekin
\end{multicols}
\end{itemize}

\indent \textit{Zure ama gizon horrekin sartuko da autoan.} (Tu madre entrará en el coche con ese hombre.)\\

\subsection{Pronombres personales: NOREKIN}

\begin{itemize}
\begin{multicols}{2}
\item (ni) $>$ nirekin
\item (hi) $>$ hirekin
\item (hura) $>$ harekin / berekin\\
\item (gu) $>$ gurekin
\item (zu) $>$ zurekin
\item (zuek) $>$ zuekin
\item (haiek) $>$ haiekin / eurekin
\end{multicols}
\end{itemize}
\indent \textit{Ni zurekin joango naiz nire lehengusuaren ezkontzara.} (Yo iré contigo a la boda de mi prima.)

\subsection{Conjunci\'on y Disyunci\'on}
\indent \textit{Amaia Mikelekin edo Xabirekin joango da Bilbora.} (Amaia irá a Bilbo con Mikel o con Xabi.)\\
\noindent "\textbf{edo}" es una conjunción coordinativa disyuntiva, es decir, indica que se debe escoger entre los elementos que une. A diferencia de la conjunción "\textbf{ala}" vista en la lección siete, "edo" se utiliza sobre todo en oraciones enunciativas.

\subsection{Expresiones temporales}
\begin{itemize}
\item gero: luego
\item bihar: mañana
\item bihar goizean: mañana por la mañana
\item etzi: pasado mañana
\item etzidamu: el día siguiente a pasado mañana
\item NOIZ? (¿Cuándo?)
\item datorren astean: la semana que viene
\item datorren hilabetean: el mes que viene
\item datorren urtean: el año que viene
\item datorren asteburuan: el fin de semana que viene
\item datorren astelehenean: el lunes que viene
\item datorren udaberrian: la primavera que viene
\end{itemize}

\section{A2 - Amaia egunero joaten da bere amaren etxera}
\indent \indent (Amaia va todos los días a casa de su madre.)\\
\noindent La forma verbal "\textbf{joaten}" hace referencia a una acción que aún no ha terminado. Esta oración da a entender una acción que se lleva a cabo con cierta frecuencia.\\
\indent \textit{Mikel astelehenero etortzen da nire bulegora.}
(Mikel viene todos los lunes a mi oficina.)\\
\indent \textit{Guk askotan jaten ditugu pastelak etxean.} (Nosotros comemos pasteles en casa a menudo.)\\

\noindent Las formas verbales que dan a entender cierta frecuencia se forman uniendo a la raíz del verbo el sufijo \textbf{-T(Z)EN}.\\
\begin{itemize}
\begin{multicols}{2}
\item joaten (joan)
\item irteten (irten)
\item egoten (egon)
\item eramaten (eraman)
\item edaten (edan)
\item jaten (jan)
\item ikusten (ikus)
\item izaten (izan)
\item etortzen (etor)
\item sartzen (sar)
\item edukitzen (eduki)
\item ekartzen (ekar)
\end{multicols}
\end{itemize}



\indent \textit{Nik \textbf{egunero} bi sagar jaten \textbf{ditut}.}
(Yo como dos manzanas todos los días.)\\
\indent \textit{Gu \textbf{urtero} herri honetara etortzen \textbf{gara}.}
(Nosotros venimos todos los años a este pueblo.)\\
\noindent Para expresar frecuencia en el pasado se utilizan los auxiliares del pasado.\\
\indent \textit{Nik \textbf{iaz} bi sagar jaten \textbf{nituen} egunero.}
(Yo el año pasado comía dos manzanas todos los días.)

\subsection{T\'erminos temporales}
\begin{itemize}
\item \textbf{Noiztik nora?} (¿de cuándo a cuándo?)\\
\indent \textit{astelehenetik ostiralera} (de lunes a viernes),\\ \textit{bostetatik zazpi eta erdietara} (de cinco a siete y media), ...
\item \textbf{Noizero?} (¿con qué frecuencia?)\\
\textit{orduoro} (cada hora / todas las horas), \textit{egunero} (todos los días), \textit{astero} (todas las semanas), \textit{asteburuoro} (todos los fines de semana), \textit{hilero} (todos los meses), \textit{goizero} (todas las mañanas), \textit{arratsaldero} (todas las tardes), \textit{gauero} (todas las noches), \textit{astelehenero} (todos los lunes), ...
\item \textbf{Zenbatetan?} (¿cuántas veces?)\\
\textit{askotan} (mucho / a menudo), \textit{gutxitan} (poco / pocas veces), \textit{batzuetan} (algunas veces), ...
\item frecuencia de de vez en cuando:\\
\textit{egunean behin} (una vez al día), \textit{astean bitan} (dos veces a la semana), \textit{hilabetean hirutan} (tres veces al mes), \textit{urtean bostetan} (cinco veces al año), ...
\end{itemize}

\subsection{Meses}
\begin{enumerate}
\begin{multicols}{2}
\item Urtarrila (Enero)
\item Otsaila (Febrero)
\item Martxoa (Marzo)
\item Apirila (Abril)
\item Maiatza (Mayo)
\item Ekaina (Junio)
\item Uztaila (Julio)
\item Abuztua (Agosto)
\item Iraila (Septiembre)
\item Urria (Octubre)
\item Azaroa (Noviembre)
\item Abendua (Diciembre)
\end{multicols}
\end{enumerate}
\end{document}