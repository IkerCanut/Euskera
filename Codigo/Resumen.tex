\documentclass[11pt, a4paper]{article}
\usepackage[left=2cm,right=2cm,top=2cm,bottom=2cm]{geometry}
\usepackage{multirow}
\usepackage{multicol}
\begin{document}
\author{Iker M. Canut}
\title{Resumen de Euskera\\}
\maketitle
\newpage

\section{A1 - Ni Amaia naiz}
\subsection{El Alfabeto}
C, Q, V, W  e  Y no son letras del alfabeto del euskera. Son letras que sólo se utilizan con nombres de otras lenguas.

\subsection{La pronunciación de s / ts, z /tz y x / tx: }
El modo de articulación en euskera de los sonidos s, z y x es parecido al de la s castellana. En cambio, el modo de articulación de los sonidos ts, tz y tx es parecido al de la ch castellana.

Por otro lado, para producir los sonidos s y ts hay que colocar la punta de la lengua en el paladar. En cambio, para crear los sonidos z y tz hay que colocar la punta de la lengua en la base de los dientes superiores. Por último, los sonidos x y tx se producen colocando el cuerpo de la lengua en el paladar.


\subsection{La pronunciación de g y j:}
La g en euskera se debe pronunciar de forma suave, igual que la g de la palabra castellana gota. En euskera, al contrario que en castellano, no se debe poner una u delante de las letras i y e.

En cuanto a la letra j, hay dos pronunciaciones posibles en euskera: una parecida a la pronunciación de la letra y de la palabra castellana ayuda; la otra, parecida a la pronunciación de la j de la palabra castellana junta. Las dos pronunciaciones son correctas, pero se tiene por mejor la primera, por ser la original del euskera. 

\subsection{La pronunciación de l y n tras i: }
En algunos dialectos del euskera, la l y n que van tras una i se pronuncian ll y ñ respectivamente.

\subsection{La pronunciación de s y ts tras i:}
Pasa algo parecido con los sonidos s y ts. En algunos dialectos se tiende a la palatalización de estos sonidos tras i, y en consecuencia, se pronuncian x y tx.

\subsection{La pronunciación de la z final en uniones de palabras:}
La letra z sufre numerosas alteraciones fonéticas en el euskera hablado. Estas alteraciones tienen lugar sobre todo con la partícula negativa EZ y con numerosas formas verbales. He aquí estas alteraciones: 
\begin{itemize}
	\item \textbf{Z + N $>$ N}: ez naiz se pronuncia \textbf{enaiz}.
	\item \textbf{Z + L $>$ L}: ez litzateke se pronuncia \textbf{elitzateke}.
	\item \textbf{Z + H $>$ $Ø$}: ez haiz se pronuncia \textbf{eaiz}.
	\item \textbf{Z + Z $>$ TZ}: ez zara se pronuncia \textbf{etzara}.
	\item \textbf{Z + D $>$ ZT}: ez da se pronuncia \textbf{ezta}.
	\item \textbf{Z + G $>$ ZK}: ez gara se pronuncia \textbf{ezkara}.
	\item \textbf{Z + B- $>$ ZP}: ez bada se pronuncia \textbf{ezpada}.
\end{itemize}

\subsection{El acento}
En euskera se tiende a acentuar la segunda sílaba empezando por la izquierda. La última sílaba también se acentúa ligeramente.

\subsection{Deklinabidea}
Igual que existen preposiciones en castellano, en euskera hay posposiciones, sufijos que suelen ir pegados a los nombres.

\subsection{El Verbo}
En euskera las formas verbales pueden ser \textbf{sintéticas} (una palabra) o \textbf{perifrásticas} (dos palabras). En las formas verbales sintéticas, toda la información verbal se concentra en una sola forma. Las formas verbales perifrásticas en cambio se dividen en más de un elemento: \textbf{verbo principal} + \textbf{verbo auxiliar}. El verbo principal puede tomar cuatro formas, por ejemplo:
\begin{itemize}
\item EROS: La raiz del verbo. Sagar bat eros ezazu!
\item EROSI: El participio. Sagar bat erosi dut.
\item EROSTEN: El imperfecto. Egunero sagar bat erosten dut.
\item EROSIKO: El futuro. Bihar sagar bat erosiko dut.
\end{itemize}
En cambio, el verbo auxiliar nos informa acerca de la persona o personas que hacen la acción que indica el verbo [nos da información sobre NOR (¿qué?), NORI (¿a quién?) y NORK (¿quién?)], el tiempo y el modo en el que transcurre dicha acción.

\subsection{Pronombres Personales}
\begin{itemize}
\begin{multicols}{2}
\item \textbf{Ni}:  1ª persona del singular
\item \textbf{Hi}:  2ª persona del singular (informal)
\item \textbf{Zu}:  2ª persona del singular (formal)
\item \textbf{Hura}:  3ª persona del singular
\item \textbf{Gu}:  1ª persona del plural
\item \textbf{Zuek}:  2ª persona del plural
\item \textbf{Haiek}:  3ª persona del plural
\end{multicols}
\end{itemize}

\subsection{Orden habitual en euskera}
\indent \indent \textbf{Sujeto + Complemento + Verbo}\\

De todas maneras, el elemento principal, el elemento inquirido o \textbf{GALDEGAIA}, se coloca delante del verbo. Por lo tanto, en euskera los elementos de la oración se ordenan según la importancia que tengan, colocándose el galdegaia delante del verbo. Los elementos restantes, si son importantes, se colocan al principio de la oración; en cambio, si no lo son, se colocan al final.\\

En las oraciones negativas se ha de intercalar una partícula que no aparece en las afirmativas: EZ (no). En consecuencia, esto afecta al orden de los elementos de la oración.\\
\indent \textbf{Sujeto EZ Verbo (Complemento)}\\

Si el verbo utilizado es perifrástico:\\
\indent \textbf{Sujeto EZ Auxiliar (Complemento) Verbo}

\subsection{IZAN}
\begin{itemize}
\begin{multicols}{2}
\item Ni naiz (yo soy)
\item Hi haiz (tú eres)
\item Hura da (él es)\\
\item Gu gara (nosotros somos)
\item Zu zara (tú eres)
\item Zuek zarete (vosotros sois)
\item Haiek dira (ellos son)
\end{multicols}
\end{itemize}

\section{A1 - Auto hau berria da}
En euskera, el \textbf{artículo determinado} tiene dos formas: \textbf{-A} (singular) y \textbf{-AK} (plural).\\
etxeA (la casa), etxeAK (las casas), etxe handiA (la casa grande), etxe handiAK (las casas grandes).\\
Nota: si el nombre termina en -a, NO se le añade otra -a.\\
Nota: los nombres propios NO llevan artículo.\\

\textbf{BAT} (uno / una) y \textbf{BATZUK} (unos / unas) son \textbf{artículos indeterminados}.\\
etxe BAT (una casa), etxe BATZUK (unas casas),\\ etxe handi BAT (una casa grande), etxe handi BATZUK (unas casas grandes).

\subsection{Pronombres Demostrativos}
Los pronombres demostrativos se colocan detrás del nombre o adjetivo
\begin{itemize}
\item hau (este, esta, esto)
\item hori (ese, esa, eso)
\item hura (aquel, aquella, aquello)
\item hauek (estos, estas)
\item horiek (esos, esas)
\item haiek (aquellos, aquellas)
\end{itemize}
etxe HAU (esta casa), etxe HAUEK (estas casas),\\ etxe handi HAU (esta casa grande), etxe handi HAUEK (estas casas grandes).\\

Si el sujeto es singular, el adjetivo llevará el artículo singular. Si es plural, el plural.\\
EtxeA handiA da (La casa es grande), Liburu HAUEK interesgarriAK dira (Estos libros son interesantes).

\subsection{Uso Correcto de los Articulos}
\textbf{EtxeA} handiA da (La casa es grande).\\
\textbf{Etxe handiA} berriA da (La casa grande es nueva).\\
En el segundo ejemplo, el sujeto está formado por un nombre y un adjetivo: \textit{etxe} y \textit{handi} forman una unidad. Cuando ésto es así, el artículo se coloca al final de la unidad.

\subsection{Algunos Adjetivos}
\begin{itemize}
\begin{multicols}{2}
\item interesante=interesgarria
\item aburrido/a=aspergarria
\item bonito/a=polita
\item feo/a=itsusia
\item rápido/a=azkarra
\item lento/a=mantsoa
\item alto/a=altua
\item bajo/a=baxua
\item grande=handia
\item pequeño/a=txikia
\item nuevo/a=berria
\item viejo/a=zaharra
\item buen/o/a=ona
\item malo/a=gaiztoa
\item largo/a=luzea
\item corto/a=motza
\item obeso/a=gizena
\item delgado/a=argala
\item listo/a=argia
\item tonto/a=tentela
\end{multicols}
\end{itemize}

\section{A1 - Nor zara zu? Ni Xabi naiz, Zarauzkoa}
\textbf{NOR?} (¿Quién? / ¿Quiénes?), \textbf{ZER?} (¿Qué?).\\
\indent Zer da hau? Hau mahaia da (¿Qué es ésto? Esto es la mesa).\\
\indent Nor zarete zuek? Gu Xabi eta Idoia gara (¿Quiénes sois vosotros? Nosotros somos Xabi e Idoia).\\

\noindent Para preguntar sobre profesiones también se utiliza el interrogativo \textbf{ZER}.\\
\indent Zer da Idoia? Idoia enpresaburua da (¿Qué es Idoia? Idoia es empresaria).\\

\noindent Cuando se quiere preguntar sobre la procedencia de alguien o algo se utiliza el interrogativo \textbf{NONGO}. En la respuesta se le añade \textbf{-koa/-koak} al nombre propio de lugar.\\
\indent Nongoa da Amaia? Amaia Oñatikoa da (¿De dónde es Amaia? Amaia es de Oñati).\\
\indent Nongoak dira politikari horiek? Politikari horiek Baionakoak dira (¿De dónde son esos políticos? Esos políticos son de Baiona).

\subsection{Agurrak (saludos)}
\begin{itemize}
\begin{multicols}{2}
\item Kaixo!: ¡Hola!
\item Aspaldiko!: ¡Cuánto tiempo sin verte!
\item Egun on!: ¡Buenos días!
\item Arratsalde on!: ¡Buenas tardes!
\item Gabon!: ¡Buenas noches!
\item Urte askotarako!: ¡Encantado!
\item Agur!: ¡Adiós!
\item Gero arte!: ¡Hasta luego!
\item Bihar arte!: ¡Hasta mañana!
\item Ikusi arte!: ¡Hasta la vista!
\end{multicols}
\item Zer moduz? Ongi, eta zu?: ¿Qué tal? Bien, ¿y tú?
\end{itemize}

\subsection{Profesiones}
\begin{itemize}
\begin{multicols}{2}
\item Arrantzalea: pescador
\item Irakaslea: profesor
\item Futbolaria: futbolista
\item Arotza: carpintero
\item Iturgina: fontanero
\item Etxekoandrea: ama de casa
\item Postaria: cartero
\item Ingeniaria: ingeniero
\item Enpresaburua: empresario
\item Alkatea: alcalde
\item Sendagilea: médico
\end{multicols}
\end{itemize}

\section{A1 - Nolakoa da autoa?}
El interrogativo \textbf{NOLAKO} se utiliza para preguntar cómo es algo.\\
\indent NolakoA da etxeA ? EtxeA handiA da (¿Cómo es la casa? La casa es grande).\\
\indent NolakoAK dira autoAK ? AutoAK azkarrAK dira (¿Cómo son los coches? Los coches son rápidos).\\
En las respuestas, los nombres y adjetivos llevan la misma marca.\\

\noindent La conjunción \textbf{BAINA} (pero) indica oposición:\\
\indent Liburua interesgarria da, baina motza (El libro es interesante pero corto).\\

\noindent Para preguntar sobre el color de algo se utiliza la siguiente pregunta: \textbf{Zein koloretakoa da … ?} (¿De qué color es?).\\
\indent  Zein koloretakoa da zerua? Zerua urdina da. (¿De qué color es el cielo? El cielo es azul.)\\

\noindent Las oraciones interrogativas generales se utilizan para preguntar sobre la veracidad o falsedad de algo, y se responden mediante sí o no. Este tipo de preguntas se pueden hacer de dos formas:
\begin{itemize}
\item Formulando las preguntas con entonación ascendente:\\
\indent  Zerua urdina da? Bai, zerua urdina da. (¿El cielo es azul? Sí, el cielo es azul.)
\item Intercalando la partícula AL:\\
\indent Zerua urdina AL da? Bai, zerua urdina da. (¿Es el cielo azul? Si, el cielo es azul.)
\end{itemize}

\subsection{Koloreak}
\begin{itemize}
\begin{multicols}{2}
\item arrosa: rosa
\item beltza: negro
\item berdea: verde
\item gorria: rojo
\item horia: amarillo
\item laranja: naranja
\item marroia: marrón
\item morea: morado
\item urdina: azul
\item zuria: blanco
\end{multicols}
\end{itemize}

\section{A1 - Amaia Donostian bizi da}
Algunos verbos en euskera son el resultado de la unión de dos palabras: un nombre + un verbo. Éste es el caso del verbo \textbf{bizi izan}. La locución verbal bizi izan es del tipo NOR.\\
\indent Amaia Donostian \textbf{bizi da}. (Amaia vive en Donostia.)\\

\noindent La oraci\'on tiene lo que llamamos \textbf{declinaci\'on}. Del mismo que en castellano existen las preposiciones, en euskera hay posposiciones que van unidas a los nombres. Para nombrar cada caso, se utiliza el interrogativo correspondiente. Aqui estamos ante el caso NON:\\
\indent \textbf{NON} bizi da Amaia? Amaia Donostian bizi da. (D\'onde vive Amaia? Amaia vive en Donostia).\\

\noindent Para responder a la pregunta NON, a los nombres propios de lugar se les a\~nade la terminaci\'on \textbf{-(E)N}. \textbf{-EN} si acaban en constonante, \textbf{-N} si acaban en vocal.\\

\subsection{Vocabulario de Lugares}
\begin{itemize}
\begin{multicols}{2}
\item auzoa: barrio
\item herrixka: aldea, pueblo pequeño
\item herria: pueblo
\item hiria: ciudad
\item eskualdea: región
\item probintzia: provincia
\end{multicols}
\end{itemize}

\section{A1 - Amaia etxean dago}
\subsection{Egon}
\begin{itemize}
\begin{multicols}{2}
\item Ni nago. (yo estoy)
\item Hi hago. (tú estás)
\item Hura dago. (él está)\\
\item Gu gaude. (nosotros estamos)
\item Zu zaude. (tú estás)
\item Zuek zaudete. (vosotros estáis)
\item Haiek daude. (ellos están)
\end{multicols}
\end{itemize}
\indent \indent eg. Ni EZ nago Zarautzen. (Yo no estoy en Zarautz.)\\
\indent eg. Zuek EZ zaudete Bermeon. (Vosotros no estáis en Bermeo.)\\

\noindent El caso NON se utiliza principalmente para indicar sitio o lugar. Para responder al caso NON, a los \textbf{nombres comunes en singular} se le debe a\~nadir la terminaci\'on \textbf{-(e)AN}: es decir:
\begin{itemize}
\item \textbf{-N} a los que terminan en vocal \textbf{-a}.  \textit{Non daude Mikel eta Xabi? Mikel eta Xabi tabernaN daude.}
\item \textbf{-AN} a los que terminan en vocal \textbf{-e, -i, -o, -u}. \textit{Non dago Amaia? Amaia etxeAN dago.}
\item \textbf{-EAN} a los que terminan en consonante. \textit{Non zaude zu? Ni komunEAN nago.}
\end{itemize}

\noindent Los adjetivos también se declinan siguiendo las leyes de arriba. La marca se usa una sola vez, es decir, \textit{Amaia etxe berriAN dago}, o por ejemplo, \textit{Mikel eta Xabi taberna zaharrEAN daude.}\\

\noindent A los \textbf{nombres en plural y en mugagabe} hay que agregarles \textbf{-ETAN} y \textbf{-(e)TAN} respectivamente.\\
\indent \textit{Haiek gure etxeetan daude.} (Ellos están en nuestras casas.)\\
\indent \textit{Haiek zenbait etxetan daude.} (Ellos están en algunas casas.)\\

\noindent A los \textbf{nombres propios de lugar}, como ya vimos, se usa el sufijo \textbf{-(e)N}.

\subsection{Demostrativos: NON}
\begin{itemize}
\begin{multicols}{2}
\item \textbf{honetan} (en este/a)
\item \textbf{horretan} (en ese/a)
\item \textbf{hartan} (en aquel/aquella)
\item \textbf{hauetan} (en estos/as)
\item \textbf{horietan} (en esos/as)
\item \textbf{haietan} (en aquellos/as)
\end{multicols}
\end{itemize}
\indent \indent \textit{Mikel zinema horretan dago.} (Mikel está en ese cine.)

\subsection{Adverbios de Lugar: NON}
\begin{itemize}
\begin{multicols}{3}
\item \textbf{hemen} (aquí)
\item \textbf{hor} (ahí)
\item \textbf{han} (allí)
\end{multicols}
\end{itemize}
\indent \indent ej. \textit{Non dago Amaia? Amaia hemen dago.} (Amaia está aquí.)\\
\indent ej. \textit{Non dago mutila? Mutila hor dago.} (El chico está ahí.)\\
\indent ej. \textit{Non bizi da Martin? Martin han bizi da.} (Martin vive allí.)\\

\subsection{Lugares}
\begin{itemize}
\begin{multicols}{2}
\item baserria: caserío
\item denda: tienda
\item egongela: sala de estar
\item eskola: escuela
\item etxea: casa
\item kalea: calle
\item komuna: cuarto de baño
\item lantokia: lugar de trabajo
\item logela: dormitorio
\item sukaldea: cocina
\item taberna: bar
\item zinema: cine
\end{multicols}
\end{itemize}

\section{A1 - Amaia haserre dago}
\noindent En euskera se utiliza el interrogativo \textbf{NOLA} para preguntar cómo está algo o alguien.\\
\indent \textit{Nola dago Amaia? Amaia haserre dago.} (¿Cómo está Amaia? Amaia está enfadada.)\\

\noindent \textbf{ALA} es la conjunción disyuntiva que se utiliza en oraciones interrogativas.\\
\indent \textit{Nola dago Amaia, haserre ala harrituta?} (¿Cómo está Amaia, enfadada o asombrada?)\\
\indent \textit{Amaia ez dago haserre. Amaia harrituta dago.}
(Amaia no está enfadada. Amaia está asombrada.)\\

\noindent \textbf{Atención}: Aunque tengan la misma traducción en castellano, no hay que confundir las estructuras \textbf{NOLA DAGO} y \textbf{NOLAKOA DA} . Mientras que NOLA vale para preguntar sobre \textbf{situaciones pasajeras}, mediante NOLAKO se pregunta sobre \textbf{situaciones permanentes}.\\
\indent ej. \textit{Nola dago Amaia? Amaia haserre dago.} (¿Cómo está Amaia? Amaia está enfadada.)\\ \indent Ahora está enfadada, pero mañana no lo estará.\\
\indent ej. \textit{Nolakoa da Amaia? Amaia polita da.} (¿Cómo es Amaia? Amaia es bonita.)\\ \indent Ahora es bonita y mañana seguirá siendo bonita.

\subsection{Como est\'as?}
\begin{itemize}
\begin{multicols}{2}
\item beldurtuta: asustado
\item bustita: mojado
\item eserita: sentado
\item etzanda: tumbado
\item garbi: limpio
\item harrituta: asombrado
\item haserre: enfadado
\item itxita: cerrado
\item maiteminduta: enamorado
\item nekatuta: cansado
\item pozik: contento
\item puskatuta: roto
\item triste: triste
\item urduri: nervioso
\item zikin: sucio
\item zutik: de pie
\end{multicols}
\end{itemize}

\section{Noren autoa da hau?}
\noindent Utilizamos el interrogativo \textbf{NOREN} cuando queremos saber quién es el dueño de algo. En euskera, a diferencia del castellano, el caso NOREN se coloca \textit{delante} de la palabra a la que acompaña\\

\indent \textit{Noren autoa da hau?} (¿El coche de quién es éste?)\\
\indent \textit{Hau aitaren autoa da.} (Éste es el coche del padre.)\\
\indent \textit{Noren giltzak dira hauek?} (¿Las llaves de quién son éstas?) \\
\indent \textit{Hauek amonaren giltzak dira.} (Éstas son las llaves de la abuela)\\

\indent \textit{Hau EZ da aitaren autoa}. (Éste no es el coche del padre.)\\

\indent \textit{Non dago Amaia?}
(¿Dónde está Amaia?)\\
\indent \textit{Amaia aitonaren ETXEAN dago.} (Amaia está en la casa del abuelo)

\subsection{Sufijos}
\begin{itemize}
\item \textbf{—AREN} para el singular: \textit{semeAREN autoa} (el coche del hijo).
\item \textbf{—EN} para el plural: \textit{semeEN autoa} (el coche de los hijos).
\item \textbf{—(r)EN} para el indeterminado: \textit{bi semeren autoa} (el coche de dos hijos)\\
\textit{zenbait mutilen autoa} (el coche de algunos chicos)
\item \textbf{-(r)EN} para los nombres propios: \textit{Amaiaren autoa, Mikelen etxea.}
\end{itemize}

\subsection{Pronombres: NOREN}
\begin{itemize}
\begin{multicols}{2}
\item Ni: nire (mi/s)
\item Hi: hire (tu/s)
\item Hura: haren / Bere (su/s)\\
\item Gu: gure (nuestro/s,nuestra/s)
\item Zu: zure (tu/s)
\item Zuek: zuen (vuestro/s, vuestra/s)
\item Haiek: haien / euren (su/s)
\end{multicols}
\end{itemize}

\noindent \textbf{bere} y \textbf{euren} son pronombres reflexivos que indican posesión:\\
\indent \textit{Amaia haren etxean dago.} (Amaia está en su casa, en casa de alguien)\\
\indent \textit{Amaia bere etxean dago.} (Amaia está en su casa, Amaia está en casa de Amaia.)

\subsection{Familia}
\begin{itemize}
\begin{multicols}{2}
\item amona/amama: abuela
\item aitona: abuelo
\item aitona-amonak: abuelos
\item biloba: nieto/a
\item ama: madre
\item aita: padre
\item gurasoak: padres
\item semea: hijo
\item alaba: hija
\item seme-alabak: hijos
\item ahizpa: hermana (de chica)
\item arreba: hermana (de chico)
\item anaia: hermano
\item neba: hermano (de chica)
\item anai-arrebak: hermanos
\item izeba: tía
\item osaba: tío
\item osaba-izebak: tíos (tío/s y tía/s)
\item iloba: sobrino/a
\item lehengusua: primo/a
\end{multicols}
\end{itemize}

\section{Norena da auto hau?}
\noindent La diferencia con el caso NOREN es la siguiente:\\
\indent ej. \textit{Noren autoA da hau?} (¿El coche de quién es éste?)\\
\indent \textit{Hau aitaren autoA da.} (Éste es el coche del padre.)\\
\indent ej. \textit{NorenA da auto hau?} (¿De quién es este coche?)\\
\indent \textit{Auto hau aitarenA da.} (Este coche es del padre.)\\

\noindent Es decir, en el caso \textbf{NOREN} se pregunta por la pertenencia de un objeto tomando como referencia al objeto, mientras que en el caso \textbf{NORENA}, se toma como referencia a la persona.\\

\noindent Para no repetir la palabra auto dos veces, se oculta la primera de las dos, uniendo el artículo al interrogativo, en este caso es \textbf{-a}.\\

\indent \textit{NorenA da bolaluma hori? Bolaluma hori nire alabarenA da.}\\
\indent \textit{NorenAK dira liburuak? Liburuak gure semeenAK dira.}\\

\indent \textit{NorenA da zapi hau? Zapi hau nireA da.} (¿De quién es este pañuelo? Este pañuelo es mío.)\\
\indent \textit{Noren zapiA da zapi hau?} (¿El pañuelo de quién es este pañuelo?)\\
\indent \indent \textit{Zapi hau nire zapiA da.} (Este pañuelo es mi pañuelo.)

\begin{multicols}{2}
\begin{itemize}
\item NireA/AK: mío/s; mía/s
\item HireA/AK: tuyo/s; tuya/s
\item HarenA/AK / BereA/AK: suyo/s; suya/s\\
\item GureA/AK: nuestro/s; nuestra/s
\item ZureA/AK: tuyo/s; tuya/s
\item ZuenA/AK: vuestro/s; vuestra/s
\item HaienA/AK / EurenA/AK: suyo/s; suya/s
\end{itemize}
\end{multicols}
\vspace{0,2cm}
\begin{multicols}{2}
\begin{itemize}
\item Hau: Honen (a/ak) (¿de éste?)
\item Hori: Horren (a/ak) (¿de ése?)
\item Hura: Haren (a/ak) (¿de aquél?)
\item Hauek: Hauen (a/ak) (¿de éstos?)
\item Horiek: Horien (a/ak) (¿de ésos?)
\item Haiek: Haien (a/ak) (¿de aquéllos?)
\end{itemize}
\end{multicols}

\subsection{Vocabulario}
\begin{itemize}
\begin{multicols}{2}
\item sare: red
\item klera: tiza
\item baloi: balón
\item zerra: sierra
\item libragailu: desatascador
\item xurgagailu: aspirador
\item gutun: carta
\item gorbata: corbata
\item xiringa: jeringuilla
\item maleta: maletín
\end{multicols}
\end{itemize}



\end{document}