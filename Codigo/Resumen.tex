\documentclass[10pt, a4paper]{article}
\usepackage[left=2.5cm, right=2.5cm, top=3.5cm, bottom=2cm]{geometry}
\usepackage{multirow}
\begin{document}
\author{Iker M. Canut}
\title{Resumen de Euskera\\}
\maketitle
\newpage

\section{1. A1 - Ni Amaia naiz}
\subsection{El Alfabeto}
C, Q, V, W  e  Y no son letras del alfabeto del euskera. Son letras que sólo se utilizan con nombres de otras lenguas.

\subsection{La pronunciación de s / ts, z /tz y x / tx: }
El modo de articulación en euskera de los sonidos s, z y x es parecido al de la s castellana. En cambio, el modo de articulación de los sonidos ts, tz y tx es parecido al de la ch castellana.

Por otro lado, para producir los sonidos s y ts hay que colocar la punta de la lengua en el paladar. En cambio, para crear los sonidos z y tz hay que colocar la punta de la lengua en la base de los dientes superiores. Por último, los sonidos x y tx se producen colocando el cuerpo de la lengua en el paladar.


\subsection{La pronunciación de g y j:}
La g en euskera se debe pronunciar de forma suave, igual que la g de la palabra castellana gota. En euskera, al contrario que en castellano, no se debe poner una u delante de las letras i y e.

En cuanto a la letra j, hay dos pronunciaciones posibles en euskera: una parecida a la pronunciación de la letra y de la palabra castellana ayuda; la otra, parecida a la pronunciación de la j de la palabra castellana junta. Las dos pronunciaciones son correctas, pero se tiene por mejor la primera, por ser la original del euskera. 

\subsection{La pronunciación de l y n tras i: }
En algunos dialectos del euskera, la l y n que van tras una i se pronuncian ll y ñ respectivamente.

\subsection{La pronunciación de s y ts tras i:}
Pasa algo parecido con los sonidos s y ts. En algunos dialectos se tiende a la palatalización de estos sonidos tras i, y en consecuencia, se pronuncian x y tx.

\subsection{La pronunciación de la z final en uniones de palabras:}
La letra z sufre numerosas alteraciones fonéticas en el euskera hablado. Estas alteraciones tienen lugar sobre todo con la partícula negativa EZ y con numerosas formas verbales. He aquí estas alteraciones: 
\begin{itemize}
	\item \textbf{Z + N $>$ N}: ez naiz se pronuncia \textbf{enaiz}.
	\item \textbf{Z + L $>$ L}: ez litzateke se pronuncia \textbf{elitzateke}.
	\item \textbf{Z + H $>$ $Ø$}: ez haiz se pronuncia \textbf{eaiz}.
	\item \textbf{Z + Z $>$ TZ}: ez zara se pronuncia \textbf{etzara}.
	\item \textbf{Z + D $>$ ZT}: ez da se pronuncia \textbf{ezta}.
	\item \textbf{Z + G $>$ ZK}: ez gara se pronuncia \textbf{ezkara}.
	\item \textbf{Z + B- $>$ ZP}: ez bada se pronuncia \textbf{ezpada}.
\end{itemize}

\subsection{El acento}
En euskera se tiende a acentuar la segunda sílaba empezando por la izquierda. La última sílaba también se acentúa ligeramente.

\subsection{Deklinabidea}
Igual que existen preposiciones en castellano, en euskera hay posposiciones, sufijos que suelen ir pegados a los nombres.

\subsection{El Verbo}
En euskera las formas verbales pueden ser \textbf{sintéticas} (una palabra) o \textbf{perifrásticas} (dos palabras). En las formas verbales sintéticas, toda la información verbal se concentra en una sola forma. Las formas verbales perifrásticas en cambio se dividen en más de un elemento: \textbf{verbo principal} + \textbf{verbo auxiliar}. El verbo principal puede tomar cuatro formas, por ejemplo:
\begin{itemize}
\item EROS: La raiz del verbo. Sagar bat eros ezazu!
\item EROSI: El participio. Sagar bat erosi dut.
\item EROSTEN: El imperfecto. Egunero sagar bat erosten dut.
\item EROSIKO: El futuro. Bihar sagar bat erosiko dut.
\end{itemize}
En cambio, el verbo auxiliar nos informa acerca de la persona o personas que hacen la acción que indica el verbo [nos da información sobre NOR (¿qué?), NORI (¿a quién?) y NORK (¿quién?)], el tiempo y el modo en el que transcurre dicha acción.

\subsection{Pronombres Personales}
\begin{itemize}
\item \textbf{Ni}:  1ª persona del singular
\item \textbf{Hi}:  2ª persona del singular (informal)
\item \textbf{Zu}:  2ª persona del singular (formal)
\item \textbf{Hura}:  3ª persona del singular
\item \textbf{Gu}:  1ª persona del plural
\item \textbf{Zuek}:  2ª persona del plural
\item \textbf{Haiek}:  3ª persona del plural
\end{itemize}

\subsection{Orden habitual en euskera}
\indent \indent \textbf{Sujeto + Complemento + Verbo}\\

De todas maneras, el elemento principal, el elemento inquirido o \textbf{GALDEGAIA}, se coloca delante del verbo. Por lo tanto, en euskera los elementos de la oración se ordenan según la importancia que tengan, colocándose el galdegaia delante del verbo. Los elementos restantes, si son importantes, se colocan al principio de la oración; en cambio, si no lo son, se colocan al final.\\

En las oraciones negativas se ha de intercalar una partícula que no aparece en las afirmativas: EZ (no). En consecuencia, esto afecta al orden de los elementos de la oración.\\
\indent \textbf{Sujeto EZ Verbo (Complemento)}\\

Si el verbo utilizado es perifrástico:\\
\indent \textbf{Sujeto EZ Auxiliar (Complemento) Verbo}

\subsection{IZAN}
\begin{itemize}
\item Ni naiz (yo soy)
\item Hi haiz (tú eres)
\item Hura da (él es)
\item Gu gara (nosotros somos)
\item Zu zara (tú eres)
\item Zuek zarete (vosotros sois)
\item Haiek dira (ellos son)
\end{itemize}
\end{document}
