\documentclass[12pt, a4paper, landscape]{article}
\usepackage[left=2.5cm, right=2.5cm, top=3.5cm, bottom=2cm]{geometry}
\usepackage{multirow}
\begin{document}
\author{Iker M. Canut}
\title{Resumen de Euskera}
\maketitle
\newpage

\tableofcontents
\newpage

\section{Aditzak}
\begin{table}[h]
\begin{center}
\begin{tabular}{|cc|c|c|c|c|c|c|c|}
\hline
&& NI & HI & HURA & GU & ZU & ZUEK & HAIEK \\
\hline
IZAN && Naiz & Haiz & Da & Gara & Zara & Zarete & Dira \\
\hline
EGON && Nago & Hago & Dago & Gaude & Zaude & Zaudete & Daude \\
\hline
EDUKI && Dau(z)kat & Dau(z)kak & Dau(z)ka & Dau(z)kagu & Dau(z)kazu & Dau(z)kazue & Dau(z)kate \\
\hline
JOAN & \small{\textit{(+IZAN)}} & Noa & Hoa & Doa & Goaz & Zoaz & Zoazte & Doaz \\
\hline
ETORRI & \small{\textit{(+IZAN)}} & Nator & Hator & Dator & Gatoz & Zatoz & Zatozte & Datoz \\
\hline
\multirow{2}{*}{UKAN} && Dut & Duk/n & Du & Dugu & Duzu & Duzue & Dute \\ && Ditut & Dituk/n & Ditu & Ditugu & Dituzu & Dituzue & Dituzte \\
\hline
\multirow{2}{*}{EKARRI} &\multirow{2}{*}{ \small{\textit{(+UKAN)}} }& Dakart & Dakark/n & Dakar & Dakargu & Dakarzu & Dakarzue & Dakarte \\ && Dakartzat & Dakartzak/n & Dakartza & Dakartzagu & Dakartzazu & Dakartzazue & Dakartzate \\
\hline
\multirow{2}{*}{ERAMAN} &\multirow{2}{*}{ \small{\textit{(+UKAN)}} }& Daramut & Daramak/n & Darama & Daramagu & Daramazu & Daramazue & Daramate \\ && Daramatzat & Daramatzak/n & Daramatza & Daramatzagu & Daramatzazu & Daramatzazue & Daramatzate \\
\hline
BAILO & \small{\textit{(+IZAN)}} & - & - & - & - & - & - & - \\
\hline
NAHI & \small{\textit{(+IZAN)}} & - & - & - & - & - & - & - \\
\hline
\end{tabular}
\end{center}
\end{table}
\newpage

\section{Erakusleak}
\begin{table}[h]
\begin{center}
\begin{tabular}{|c|c|c|c|c|c|c|}
\hline
& HAU & HORI & HURA & HAUEK & HORIEK & HAIEK \\
\hline
NON & Honetan & Horretan & Hartan & Hauetan & Horietan & Haietan \\
\hline
NOREN & Honen & Horren & Haren & Hauen & Horien & Haien \\
\hline
NORK & Honek & Horrek & Hark & Hauek & Horiek & Haiek \\
\hline
NORA & Honetara & Horretara & Hartara & Hauetara & Horietara & Haietara \\ 
\hline
NORENGANA & Honengana & Horrengana & Harengana & Hauengana & Horiengana & Haiengana \\
\hline
NONDIK & Honetatik & Horretatik & Hartatik & Hauetatik & Horietatik & Haietatik \\
\hline
NORENTZAT & Honentzat & Horrentzat & Harentzat & Hauentzat & Horientzat & Haientzat \\
\hline
\end{tabular}
\end{center}
\end{table}

\section{Pertsona-Izenordainak}
\begin{table}[h]
\begin{center}
\begin{tabular}{|c|c|c|c|c|c|c|c|}
\hline
NOR & NI & HI & HURA & GU & ZU & ZUEK & HAIEK \\ 
\hline
NORI & Niri & Hiri & Hari & Guri & Zuri & Zuei & Haiei \\
\hline
NORK & Nik & Hik & Hark & Guk & Zuk & Zuek & Haiek \\
\hline
NOREN & Nire & Hire & Haren/Bere & Gure & Zure & Zuen & Haien/Euren \\
\hline
NORENGANA & Niregana & Hiregana & Harengana/Beregana & Guregana & Zuregana & Zuengana & Haiengana/Eurengana \\
\hline
NORENTZAT & Niretzat & Hiretzat & Harentzat/Beretzat & Guretzat & Zuretzat & Zuentzat & Haientzat/Eurentzat \\
\hline
\end{tabular}
\end{center}
\end{table}

\newpage
\section{Leku-Adberbioak}
\begin{table}[h]
\begin{center}
\begin{tabular}{|c|c|c|c|}
\hline
NON & Hemen & Hor & Han \\
\hline
NONDIK & Hemendik & Hortik & Handik \\
\hline
\end{tabular}
\end{center}
\end{table}

\section{Atzizkiak}
\begin{table}[h]
\begin{center}
\begin{tabular}{|c|c|c|c|c|}
\hline
& Singularrerako & Pluralerako & Mugagaberako & Leku-izen bereziekin\\
\hline
NON & (e)AN & ETAN & (e)TAN & - \\
\hline
NOREN & AREN & EN & (r)EN & (r)EN \\
\hline
NORK & AK & EK & (e)K & (e)K \\
\hline
NORA & (e)RA & ETARA & (e)TARA & (e)RA \\
\hline
NORENGANA & ARENGANA & ENGANA & (r)ENGANA & - \\
\hline
NONDIK & (e)TIK & ETATIK & (e)TATIK & (e)TIK // -(L/N)DIK \\
\hline
NORENTZAT & ARENTZAT & ENTZAT & (r)ENTZAT & - \\
\hline

\end{tabular}
\end{center}
\end{table}
\newpage

\section{Zenbakiak}
\begin{table}[h]
\begin{center}
\begin{tabular}{|c|c||c|c|}
\hline
0 & Zero & 20 & Hogei\\
\hline
1 & Bat & 40 & Berrogei\\
\hline
2 & Bi & 60 & Hirugogei\\
\hline
3 & Hiru & 80 & Laurogei\\
\hline
4 & Lau & 90 & Larogeita hamar\\
\hline
5 & Bost & 100 & Ehun\\
\hline
6 & Sei & 200 & Berrehun\\
\hline
7 & Zazpi & 300 & Hirurehun\\
\hline
8 & Zortzi & 400 & Laurehun\\
\hline
9 & Bederatzi & 500 & Bostehun\\
\hline
10 & Hamar & 600 & Seiehun\\
\hline
11 & Hamaika & 700 & Zazpiehun\\
\hline
12 & Hamabi & 800 & Zortziehun\\
\hline
13 & Hamahiru & 900 & Bederatziehun\\
\hline
14 & Hamalau & 1.000 & Mila\\
\hline
15 & Hamabost & 2.000 & Bi mila\\
\hline
16 & Hamasei & 10.000 & Hamar mila\\
\hline
17 & Hamazazpi & 500.000 & Bostehun mila\\
\hline
18 & Hamazortzi & 1.000.000 & Milioi bat\\
\hline
19 & Hemeretzi & 1.000.000.000 & Miliar bat\\
\hline
\end{tabular}
\quad
\begin{tabular}{|c|c|}
\hline
1:00 & Ordu bata\\
\hline
1:05 & Ordu bata eta bost\\
\hline
2:00 & Ordu biak\\
\hline
3:00 & Hirurak\\
\hline
3:15 & Hiru eta laurdenak\\
\hline
4:00 & Laurak\\
\hline
5:00 & Bostak\\
\hline
5:25 & Bostak eta hogeita bost\\
\hline
6:00 & Seiak\\
\hline
6:30 & Sei eta erdiak\\
\hline
6:35 & Zazpiak hogeita bost gutxi\\
\hline
7:00 & Zazpiak\\
\hline
8:00 & Zortziak\\
\hline
8:45 & Bederatziak laurden gutxi\\
\hline
9:00 & Bederatziak\\
\hline
9:50 & Hamarrak hamar gutxi\\
\hline
10:00 & Hamarrak\\
\hline
11:00 & Hamaikak\\
\hline
12:00 & Hamabiak\\
\hline
\end{tabular}
\quad
\begin{tabular}{|c|c|}
\hline
:00 & -etan \\
\hline
:05 & -ean \\
\hline
:10 & -ean \\
\hline
:15 & -etan \\
\hline
:20 & -an \\
\hline
:25 & -ean \\
\hline
:30 & -etan \\
\hline
:35 & -tan \\
\hline
:40 & -tan \\
\hline
:45 & -tan \\
\hline
:50 & -tan \\
\hline
:55 & -tan \\
\hline
\end{tabular}
\end{center}
\end{table}
\newpage



\section{Preguntas}

\begin{table}[h]
\begin{center}
\begin{tabular}{||cc}
NOR...? & Quién...? \\
ZER...? & Qué...? \\
NON...? & Dónde...? \\
NONGOA...? & De dónde...? \\
ZEIN KOLORETAKO...? & De qué color...? \\
NOLAKO...? & Cómo es...? \\
NOLA...? & Cómo está...? \\
NOREN...? & De quién...? (Objeto $\to$ Pertenencia)\\
NORENA/K...? & De quién...? (Persona $\to$ Pertenencia) \\
ZEREN...? & De qué...? \\
ZENBAT...? & Cuánto...? \\
NORA...? & A dónde...? \\
NORENGANA...? & Donde quién...? \\
NONDIK...? & De dónde...? \\
NORENTZAT...? & Para quién...? \\
NOIZ...? & Cuando...? \\
ZER ORDUTAN...? & A qué hora...? \\
ZER ORDU...? & Que hora...?\\
ZENBAT BAILO DU...?  & Cuánto vale...?\\
ZENBAT DA...? & Cuánto es...?\\
ZENBATEAN DAGO...? & A cuánto está...?
\end{tabular}
\end{center}
\end{table}

\end{document}