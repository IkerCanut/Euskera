\documentclass[12pt]{article}
\usepackage[left=1.8cm, right=2.5cm, top=4cm]{geometry}
\begin{document}
\author{Iker M. Canut}
\title{Resumen de Euskera}
\maketitle
\newpage

\tableofcontents
\newpage

\section{Aditzak}
\begin{table}[h]
\begin{center}
\begin{tabular}{|c|c|c|c|c|c|c|c|}
\hline
& NI & HI & HURA & GU & ZU & ZUEK & HAIEK \\
\hline
IZAN & Naiz & Haiz & Da & Gara & Zara & Zarete & Dira \\
\hline
EGON & Nago & Hago & Dago & Gaude & Zaude & Zaudete & Daude \\
\hline
EDUKI & Dau(z)kat & Dau(z)kak & Dau(z)ka & Dau(z)kagu & Dau(z)kazu & Dau(z)kazue & Dau(z)kate \\
\hline
JOAN & Noa & Hoa & Doa & Goaz & Zoaz & Zoazte & Doaz \\
\hline
ETORRI & Nator & Hator & Dator & Gatoz & Zatoz & Zatozte & Datoz \\
\hline
\end{tabular}
\end{center}
\end{table}
\newpage

\section{Erakusleak}
\begin{table}[h]
\begin{center}
\begin{tabular}{|c|c|c|c|c|c|c|}
\hline
& HAU & HORI & HURA & HAUEK & HORIEK & HAIEK \\
\hline
NON & Honetan & Horretan & Hartan & Hauetan & Horietan & Haietan \\
\hline
NOREN & Honen & Horren & Haren & Hauen & Horien & Haien \\
\hline
NORK & Honek & Horrek & Hark & Hauek & Horiek & Haiek \\
\hline
NORA & Honetara & Horretara & Hartara & Hauetara & Horietara & Haietara \\ 
\hline
NORENGANA & Honengana & Horrengana & Harengana & Hauengana & Horiengana & Haiengana \\
\hline
NONDIK & Honetatik & Horretatik & Hartatik & Hauetatik & Horietatik & Haietatik \\
\hline
\end{tabular}
\end{center}
\end{table}
\newpage

\section{Pertsona-Izenordainak}
\begin{table}[h]
\begin{center}
\begin{tabular}{|c|c|c|c|c|c|c|c|}
\hline
NOR & NI & HI & HURA & GU & ZU & ZUEK & HAIEK \\ 
\hline
NORI & Niri & Hiri & Hari & Guri & Zuri & Zuei & Haiei \\
\hline
NORK & Nik & Hik & Hark & Guk & Zuk & Zuek & Haiek \\
\hline
NOREN & Nire & Hire & Haren/Bere & Gure & Zure & Zuen & Haien/Euren \\
\hline
\end{tabular}
\end{center}
\end{table}
\newpage

\section{Leku-Adberbioak}
\begin{table}[h]
\begin{center}
\begin{tabular}{|c|c|c|c|}
\hline
NON & Hemen & Hor & Han \\
\hline
NONDIK & Hemendik & Hortik & Handik \\
\hline
\end{tabular}
\end{center}
\end{table}
\newpage

\section{Atzizkiak}
\begin{table}[h]
\begin{center}
\begin{tabular}{|c|c|c|c|c|}
\hline
& Singularrerako & Pluralerako & Mugagaberako & Leku-izen bereziekin\\
\hline
NON & (e)AN & ETAN & (e)TAN & - \\
\hline
NOREN & AREN & EN & (r)EN & (r)EN \\
\hline
NORK & AK & EK & (e)K & (e)K \\
\hline
NORA & (e)RA & ETARA & (e)TARA & (e)RA \\
\hline
NORENGANA & ARENGANA & ENGANA & (r)ENGANA & - \\
\hline
NONDIK & (e)TIK & ETATIK & (e)TATIK & (e)TIK // -(L/N)DIK \\
\hline

\end{tabular}
\end{center}
\end{table}
\newpage

\section{Preguntas}

\begin{table}[h]
\begin{center}
\begin{tabular}{cc}
NOR...? & Quién...? \\
ZER...? & Qué...? \\
NON...? & Dónde...? \\
NONGOA...? & De dónde...? \\
ZEIN KOLORETAKO...? & De qué color...? \\
NOLAKO...? & Cómo es...? \\
NOLA...? & Cómo está...? \\
NOREN...? & De quién...? (Objeto $\to$ Pertenencia)\\
NORENA/K...? & De quién...? (Persona $\to$ Pertenencia) \\
ZEREN...? & De qué...? \\
ZENBAT...? & Cuánto...? \\
NORA...? & A dónde...? \\
NOIZ...? & Cuándo...? \\
NORENGANA...? & Donde quién...? \\
NONDIK...? & De dónde...? \\
\end{tabular}
\end{center}
\end{table}

\end{document}